\documentclass[12pt,oneside]{fithesis2}

% ===== LOADING PACKAGES =====
% language settings, main documnet language last
\usepackage[english]{babel}
% enabling new fonts support (nicer)
\usepackage{lmodern}
% setting input encoding
\usepackage[utf8]{inputenc}
% setting output encoding
\usepackage[T1]{fontenc}
% fithesis2 requires csquotes
\usepackage{csquotes}
% set page margins
\usepackage[top=3.5cm, bottom=3cm, left=2.4cm, right=2.4cm]{geometry}
% package to make bullet list nicer
\usepackage{enumitem}
% math symbols and environments
\usepackage{mathtools}
% packages for complex tables
\usepackage{tabularx}
\usepackage{multirow}
\usepackage{siunitx}
\usepackage{dcolumn}
\usepackage{array}
% enable page rotation
\usepackage{pdflscape}
% generating hyperlinks in document
\usepackage{url}
\usepackage[plainpages=false,pdfpagelabels,colorlinks=true]{hyperref}

% ===== MAIN DOCUMENT SETTINGS =====
% adjusting hyphenation penalties
\tolerance=10000
\hyphenpenalty=500
% FI THESIS settings
\thesistitle{Usage of evolvable circuit \\for statistical testing \\of randomness}
\thesissubtitle{Bachelor thesis}
\thesisstudent{Martin Ukrop}
\thesiswoman{false}
\thesisfaculty{fi}
\thesisyear{spring 2013}
\thesisadvisor{RNDr.\ Petr Švenda,\ Ph.D.}
\thesislang{en}
% set correct spacing in itemize
\setitemize{noitemsep,topsep=0pt,parsep=0pt,partopsep=0pt}

% new commands for results table headers
\newcommand{\m}[1]{\multirow{2}*{#1}}
\newcommand{\rotatedHeader}[2][l]{\rotatebox{90}{\begin{tabular}[#1]{@{}l}#2\end{tabular}}}
\newcommand{\resultsTable}[1]{%
\newcolumntype{C}{>{\centering\arraybackslash}X}%
\begin{tabularx}{\textwidth}{|r*{3}{||S[table-format=2.1]|S[table-format=3.0]|C}|} \hline
\multirow{1}*{\raisebox{-\height-0.3cm}{\rotatebox{90}{\# of rounds}}} & \multicolumn{9}{c|}{IV and key reinitialization} \\ \cline{2-10}
& \multicolumn{3}{c||}{once for run} & 
\multicolumn{3}{c||}{for each test set} & 
\multicolumn{3}{c|}{for each test vector} \\ \cline{2-10}
& \multicolumn{1}{c|}{\rotatedHeader{Dieharder\\(x/20)}} & \multicolumn{1}{c|}{\rotatedHeader{\textsc{Sts Nist}\\(x/162)}} &  \rotatedHeader{EACirc\\(\textsc{aam})}
& \multicolumn{1}{c|}{\rotatedHeader{Dieharder\\(x/20)}} & \multicolumn{1}{c|}{\rotatedHeader{\textsc{Sts Nist}\\(x/162)}} & \rotatedHeader{EACirc\\(\textsc{aam})}
& \multicolumn{1}{c|}{\rotatedHeader{Dieharder\\(x/20)}} & \multicolumn{1}{c|}{\rotatedHeader{\textsc{Sts Nist}\\(x/162)}} & \rotatedHeader{EACirc\\(\textsc{aam})} \\ \hline \hline
#1
\end{tabularx}}

% ===== BEGIN DOCUMENT =====
\begin{document}

\FrontMatter
\ThesisTitlePage

\begin{ThesisDeclaration}
\DeclarationText
\AdvisorName
\end{ThesisDeclaration}

\begin{ThesisThanks}
Thanks will be here.
\end{ThesisThanks}

\begin{ThesisAbstract}
Abstract will be here.
\end{ThesisAbstract}

\begin{ThesisKeyWords}
Keywords will be here.
\end{ThesisKeyWords}

\MainMatter
\tableofcontents
\chapter{Introduction}
\label{chap:intro}
Text ...

\chapter{Statistical randomness testing}
\label{chap:stat-rand-testing}

\begin{itemize}
\item idea: statistic (maths) -> test
\item fast
\item universal
\item usage: assess quality of (pseudo)random data, ???
\end{itemize}

\section{Statistical Test Suite by NIST}
\label{sec:sts-nist}

\begin{itemize}
\item nist standard
\item short description (?)
\end{itemize}

\section{Diehard battery of tests}
\label{sec:diehard}

\begin{itemize}
\item author
\item one of the first and most-well known in those years
\item old, but still considered "gold standard" along with sts-nist
\item short description of tests (?)
\end{itemize}

\section{Dieharder: A Random Number Test Suite}
\label{sec:dieharder}

\begin{itemize}
\item framework idea
\item progress: diehard -> sts-nist -> new
\end{itemize}

\section{Limits and disadvantages of statistical testing}
\label{sec:limits-stat-testing}

\begin{itemize}
\item idea -> test (idea is always the predecessor)
\item check only one specific property
\item can only rarely be adapted to specific situation
\item results interpretation (what is wrong?)
\end{itemize}

\chapter{Evolution based randomness testing}
\label{chap:evo-based-testing}
\begin{itemize}
\item general description of GA
\item idea behind EACirc
\item previous evolution of EACirc (SensorSim -> bc, mgr -> today)
\item capabilities of EACirc
\begin{itemize}
\item general object model (+picture)
\item separate modules for projects
\item separate modules for evaluators
\item guaranteed bit-reproducibility
\item computation recommencing (state, ...)
\item static checker for pregenerated test vectors
\end{itemize}
\item EACirc is wider project beyond the scope of this thesis, thus project evolution, some parts are being redesigned
\end{itemize}

\chapter{Experiment settings and results}
\label{chap:settings-results}

\begin{itemize}
\item general evolution settings
\item main goal: finding strong distinguisher (over 99\% for 50 consecutive generations)
\item displayed average stable generation across 30 independent runs \\
(stable = fitness over $99\%$ for at least next 50 test sets)
\item if none stable generation was found, average average maximum fitness after test vector change is displayed in parentheses.
\item statistical batteries STS-NIST and Dieharder for reference
\item 250 MB of data used, same files for Dieharder and STS-NIST
\item STS-NIST settings (lenghts, 2 test omitted)
\item each test run 100 times on 1\,000\,000 bits
\item some runs had problems with tests RandomExcursions and RandomExcursionsVariant, these tests are not included in the result
\item STS-NIST results interpretation (scores 0, 1)
\item Dieharder settings
\item test corresponding to original Diehard (except for Diehard sums test)
\item each test run once, length of the stream decided by test
\item Dieharder results interpretation (scores 0, 0.5, 1)
\item displayed number of tests passed out of total (pass=1, weak=0.5, fail=0)
\end{itemize}

\chapter{Control distinguishers}
\label{chap:distinguish-control}

\begin{itemize}
\item introduction -- the need of reference numbers before analysis
\item we need to define what does it mean "indistinguishable" in our setting
\item we use quantum random data from Humboldt Universitat and Quantum random bit generator service as a standard for randomness
\end{itemize}

\section{Looking for non-randomness in quantum random data}
\label{sec:control-random-random}

\begin{itemize}
\item trying to distinguish quantum random data from quantum random data \\ => we presume to fail
\item using random data from Quantum random bit generator service
\item statistical batteries: data are random (Dieharder: 20/20, STS NIST: 188/188)
\item evolution: no stable distinguisher found, AAM of 0.52 (differences in various runs in 3rd or 4th decimal place)
\item presumption: dependence on test set size and population size
\item presumption confirmed (\autoref{tab:random-set-size-change}), AAM decreases with smaller population and bigger test set size
\end{itemize}

\begin{table}[htb]
\centering
\renewcommand{\arraystretch}{1.2}
\newcolumntype{C}{>{\centering\arraybackslash}X}
\begin{tabularx}{\textwidth}{|c|r||*{6}{C|}} \cline{3-8}
\multicolumn{2}{c||}{} & \multicolumn{6}{c|}{number of test vector in a set} \\ \cline{3-8}
\multicolumn{2}{c||}{} & 200 & 500 & 1000 & 2000 & 5000 & 10\,000 \\ \hline \hline
\multirow{5}*{\rotatedHeader{individuals \\ in population}}
& 5 & -- & -- & (0.509) & - & - & - \\ \cline{2-8}
& 10 & -- & -- & (0.514) & - & - & - \\ \cline{2-8}
& 20 & (0.544) & (0.527) & (0.520) & (0.514) & (0.509) & (0.506) \\ \cline{2-8}
& 50 & - & - & (0.526) & - & - & - \\ \cline{2-8}
& 100 & - & - & (0.530) & - & - & - \\ \hline
\end{tabularx}
\renewcommand{\arraystretch}{1.0}
\caption{Dependence of AAM on population size and test vector set size.}
\label{tab:random-set-size-change}
\end{table}

\section{Distinguishing quantum random data from different sources}
\label{sec:control-germany-croatia}

\begin{itemize}
\item distinguishing streams of quantum random data from Humboldt University and streams of quantum random data from Ruđer Bošković Institute
\begin{itemize}
\item Quantum Random Bit Generator Service, Centre for Informatics and Computing, Ruđer Bošković Institute, Zagreb, Croatia
\item Quantum Random Number Generator Service, Department of Physics, Humboldt University, Berlin, Germany
\end{itemize}
\item 6 files of 5 MB from each source
\item fixed initial reading offsets as (0,0) \\ => same data from given file in each run
\item looking for distinguisher for each pair
\item interpretation of results (\autoref{tab:control-germany-croatia}):
\begin{itemize}
\item data from both sources are equally random for our purposes
\item there is no single statistically different stream in these \\=>they can be used interchangeably 
\end{itemize}
\end{itemize}

\begin{table}[htb]
\centering
\renewcommand{\arraystretch}{1.2}
\newcolumntype{C}{>{\centering\arraybackslash}X}
\begin{tabularx}{\textwidth}{|c|r||*{6}{C|}} \cline{3-8}
\multicolumn{2}{c||}{} & \multicolumn{6}{c|}{QRBG service (Ruđer Bošković Institute, Croatia)} \\ \cline{3-8}
\multicolumn{2}{c||}{} & stream 1 & stream 2 & stream 3 & stream 4 & stream 5 & stream 6 \\ \hline \hline
\multirow{6}*{\rotatedHeader{QRNG service \\(HU, Germany)}}
& stream 1 & (0.521) & (0.520) & (0.520) & (0.519) & (0.519) & (0.519) \\ \cline{2-8}
& stream 2 & (0.158) & (0.519) & (0.520) & (0.520) & (0.520) & (0.519) \\ \cline{2-8}
& stream 3 & (0.519) & (0.522) & (0.519) & (0.520) & (0.519) & (0.519) \\ \cline{2-8}
& stream 4 & (0.520) & (0.520) & (0.519) & (0.518) & (0.519) & (0.519) \\ \cline{2-8}
& stream 5 & (0.519) & (0.520) & (0.519) & (0.518) & (0.520) & (0.520) \\ \cline{2-8}
& stream 6 & (0.520) & (0.519) & (0.520) & (0.520) & (0.519) & (0.519) \\ \hline
\end{tabularx}
\renewcommand{\arraystretch}{1.0}
\caption{Distinguishing binary quantum random streams from independent sources.}
\label{tab:control-germany-croatia}
\end{table}

\section{Uncompressed audio streams}
\label{sec:distinguishing-audio}

\begin{itemize}
\item 12 files
\begin{itemize}
\item 3 quantum random files
\item 3 noise files (white, pink, brown)
\item 3 noise files with oscillating volume (2 cycles in given 30 seconds)
\item 3 samples of transcendental khaoblack metal music
\end{itemize}
\item each file is 30sec of uncompressed WAV audio (5.3MB)
\item evolving distinguisher for each pair
\item interpretation of results (\autoref{tab:uncompressed-audio}):
\begin{itemize}
\item quantum random stream are undistinguishable (we already know)
\item random stream 2 is more "noise-like" \\ => reminds us to use random data cautiously, and use statistics to evaluate results
\item uncompressed WAV audio is quite easily distinguishable from random stream of data (generally over 80\%)
\item different types of noise can be quite successfully distinguished from one another (generally over 70\%)
\item it is difficult to distinguish noise from its oscillating version (around 58\%)
\item when using oscillating noise for distinguishing, fitness is not oscillating \\ => volume is not statistically important in these sample noise files
\item given samples of metal music can be quite successfully distinguished from noise
\item given samples of metal music nearly cannot be distinguished from one another
\end{itemize}
\item most of the runs have slow rising tendency in fitness \\ => if more generations, the average maximum value might be slightly higher
\end{itemize}

\begin{landscape}
\begin{table}[p]
\centering
\newsavebox{\temp}
\newcolumntype{C}{>{\centering \begin{lrbox}{\temp} \arraybackslash}X<{\end{lrbox} \m{\unhbox\temp} \arraybackslash}}
\begin{tabularx}{22cm}{|c|>{\raggedright\arraybackslash}p{2.5cm}*{4}{||C|C|C}|} \cline{3-14}

\multicolumn{2}{l||}{} & \multicolumn{3}{c||}{random streams} & \multicolumn{3}{c||}{noise} &
\multicolumn{3}{c||}{noise (oscillating)} & \multicolumn{3}{c|}{metal music} \\ \cline{3-14}

\multicolumn{2}{l||}{} &  
\multicolumn{1}{c|}{\rotatedHeader{random\\stream 1}} & 
\multicolumn{1}{c|}{\rotatedHeader{random\\stream 2}} & 
\multicolumn{1}{c||}{\rotatedHeader{random\\stream 3}} & 
\multicolumn{1}{c|}{\rotatedHeader{white noise}} & 
\multicolumn{1}{c|}{\rotatedHeader{pink noise}} & 
\multicolumn{1}{c||}{\rotatedHeader{brown noise}} & 
\multicolumn{1}{c|}{\rotatedHeader{white noise\\(oscillating)}} & 
\multicolumn{1}{c|}{\rotatedHeader{pink noise\\(oscillating)}} & 
\multicolumn{1}{c||}{\rotatedHeader{brown noise\\(oscillating)}} & 
\multicolumn{1}{c|}{\rotatedHeader{metal music\\(sample 1)}} & 
\multicolumn{1}{c|}{\rotatedHeader{metal music\\(sample 2)}} & 
\multicolumn{1}{c|}{\rotatedHeader{metal music\\(sample 3)}} \\ \cline{3-14} \hline \hline

\multirow{3}{*}[-20pt]{\rotatedHeader{random}} &
random stream 1 & 
n/a & (0.52) & (0.52) & (0.91) & (0.96) & (0.97) & (0.87) & (0.93) & (0.95) & (0.79) & (0.84) & (0.88) \\ \cline{2-14}
& random stream 2 &
(0.52) & n/a & (0.52) & (0.82) & (0.85) & (0.83) & (0.86) & (0.91) & (0.96) & (0.89) & (0.85) & (0.87) \\ \cline{2-14}
& random stream 3 & 
(0.52) & (0.52) & n/a & (0.94) & (0.96) & (0.95) & (0.95) & (0.96) & (0.91) & (0.82) & (0.88) & (0.87) \\ \hline \hline
\multirow{3}{*}[-20pt]{\rotatedHeader{noise}} & 
white noise (constant) &
(0.91) & (0.82) & (0.94) & n/a & (0.71) & (0.84) & (0.59) & (0.80) & (0.96) & (0.78) & (0.80) & (0.79) \\ \cline{2-14}
& pink noise (constant) &
(0.96) & (0.85) & (0.96) & (0.71) & n/a & (0.72) & (0.70) & (0.63) & (0.68) & (0.70) & (0.74) & (0.75) \\ \cline{2-14}
& brown noise (constant) &
(0.97) & (0.83) & (0.95) & (0.84) & (0.72) & n/a & (0.80) & (0.65) & (0.53) & (0.80) & (0.73) & (0.83) \\ \hline \hline
\multirow{3}{*}[-10pt]{\rotatedHeader{noise (osc.)}} & 
white noise (oscillating) &
(0.87) & (0.86) & (0.95) & (0.59) & (0.70) & (0.80) & n/a & (0.80) & (0.84) & (0.80) & (0.80) & (0.80) \\ \cline{2-14}
& pink noise (oscillating) &
(0.93) & (0.91) & (0.96) & (0.80) & (0.63) & (0.65) & (0.80) & n/a & (0.62) & (0.81) & (0.84) & (0.82) \\ \cline{2-14}
& brown noise (oscillating) &
(0.95) & (0.96) & (0.91) & (0.96) & (0.68) & (0.53) & (0.84) & (0.62) & n/a & (0.75) & (0.84) & (0.86) \\ \hline \hline
\multirow{3}{*}[-5pt]{\rotatedHeader{metal music}} & 
metal music (sample 1) &
(0.79) & (0.89) & (0.82) & (0.78) & (0.70) & (0.80) & (0.80) & (0.81) & (0.75) & n/a & (0.54) & (0.54) \\ \cline{2-14}
& metal music (sample 2) &
(0.84) & (0.85) & (0.88) & (0.80) & (0.74) & (0.73) & (0.80) & (0.84) & (0.84) & (0.54) & n/a & (0.57) \\ \cline{2-14}
& metal music (sample 3) &
(0.88) & (0.87) & (0.87) & (0.97) & (0.75) & (0.83) & (0.80) & (0.82) & (0.86) & (0.54) & (0.57) & n/a \\ \cline{1-14}
\end{tabularx}
\caption{Distinguishing random streams and uncompressed audio (noise, oscillating noise, metal music).}
\label{tab:uncompressed-audio}
\end{table}
\end{landscape}

\chapter{Distinguishing cipher outputs from random stream}
\label{chap:distinguish-cipher}

\begin{itemize}
\item introduction, idea, running EACirc along with statistical batteries
\item stream ciphers from eStream competition
\end{itemize}

\section{Stream ciphers used}
\label{sec:estream-ciphers}

\begin{itemize}
\item ciphers except for ?? (why??)
\item from last phase
\item those that could be limited in rounds are tested in weaker variant as well
\item differences from Metej Pristak thesis
\end{itemize}

\section{Generating binary stream from stream ciphers}
\label{sec:estream-settings}

\begin{itemize}
\item cipher modes (iv+key initialization frequency)
\item case of LEX (not weakening the cipher, only making shorter output)
\item case of TSC (producing binary stream of 0 for 1-8 rounds) => problems in 3 Dieharder tests
\end{itemize}

\section{Results interpretation}
\label{sec:estream-results}

\begin{itemize}
\item ???
\item more or less as statistical batteries
\item dieharder better in some case than STS-NIST (is newer and some tests are redesigned)
\item statistical tests has much more input data compared to EACirc
\item using evolved distinguisher is quick
\end{itemize}

\begin{table}[htb]
\centering
\resultsTable{
1 & 0.0 & 0 & $n=2681$ & 0.0 & 0 & (0.85) & 0.0 & 5 & $n=1431$ \\ \hline
2 & 0.5 & 0 & (0.54) & 1.0 & 0 & (0.54) & 15.5 & 146 & (0.52) \\ \hline
3 & 1.0 & 0 & (0.53) & 1.0 & 0 & (0.53) & 15.0 & 160 & (0.52) \\ \hline
4 & 3.5 & 79 & (0.52) & 3.0 & 78 & (0.52) & 20.0 & 160 & (0.52) \\ \hline
5 & 4.5 & 79 & (0.52) & 3.5 & 91 & (0.52) & 17.5 & 161 & (0.52) \\ \hline
6 & 19.0 & 158 & (0.52) & 19.0 & 159 & (0.52) & 18.0 & 162 & (0.52) \\ \hline
7 & 18.5 & 162 & (0.52) & 19.0 & 161 & (0.52) & 20.0 & 161 & (0.52) \\ \hline \hline
8 & 20.0 & 162 & (0.52) & 20.0 & 159 & (0.52) & 19.0 & 161 & (0.52) \\ \hline
}
\caption{Random distinguishers for Decim ciphertext.}
\label{tab:estream-decim}
\end{table}

\begin{table}[htb]
\centering
\resultsTable{
1 & 20.0 & 162 & (0.52) & 20.0 & 161 & (0.52) & 18.0 & 162 & (0.52) \\ \hline \hline
4 & 20.0 & 162 & (0.52) & 20.0 & 162 & (0.52) & 20.0 & 162 & (0.52) \\ \hline
}
\caption{Random distinguishers for FUBUKI ciphertext.}
\label{tab:estream-fubuki}
\end{table}

\begin{table}[htb]
\centering
\resultsTable{
1 & 0.0 & 0 & $n=221$ & 0.0 & 0 & (0.67) & 18.5 & 162 & (0.52) \\ \hline
2 & 0.0 & 0 & $n=471$ & 0.5 & 0 & (0.66) & 20.0 & 162 & (0.52) \\ \hline
3 & 19.5 & 160 & (0.52) & 20.0 & 162 & (0.52) & 20.0 & 162 & (0.52) \\ \hline \hline
13 & 20.0 & 162 & (0.52) & 20.0 & 161 & (0.52) & 19.5 & 162 & (0.52) \\ \hline
}
\caption{Random distinguishers for Grain ciphertext.}
\label{tab:estream-grain}
\end{table}

\begin{table}[htb]
\centering
\resultsTable{
1 & 20.0 & 162 & (0.52) & 20.0 & 162 & (0.52) & 20.0 & 162 & (0.52) \\ \hline \hline
10 & 20.0 & 160 & (0.52) & 20.0 & 162 & (0.52) & 20.0 & 162 & (0.52) \\ \hline
}
\caption{Random distinguishers for Hermes ciphertext.}
\label{tab:estream-hermes}
\end{table}

\begin{table}[htb]
\centering
\resultsTable{
1 & 0.0 & 0 & $n=148$ & 0.0 & 0 & $n=7274$ & 3.0 & 1 & $n=154$ \\ \hline
2 & 4.0 & 1 & $n=221$ & 4.0 & 1 & $n=304$ & 3.5 & 1 & $n=254$ \\ \hline
3 & 0.5 & 1 & $n=378$ & 3.5 & 1 & $n=491$ & 4.0 & 1 & $n=361$ \\ \hline
4 & 20.0 & 162 & (0.52) & 19.5 & 162 & (0.52) & 20.0 & 161 & (0.52) \\ \hline \hline
10 & 19.5 & 162 & (0.52) & 19.5 & 160 & (0.52) & 20.0 & 160 & (0.52) \\ \hline
}
\caption{Random distinguishers for LEX ciphertext.}
\label{tab:estream-lex}
\end{table}

\begin{table}[htb]
\centering
\resultsTable{
1 & 5.5 & 1 & (0.87) & 8.5 & 1 & (0.67) & 17.5 & 161 & (0.52) \\ \hline
2 & 5.5 & 1 & (0.87) & 7.0 & 1 & (0.67) & 19.5 & 162 & (0.52) \\ \hline
3 & 20.0 & 162 & (0.52) & 20.0 & 162 & (0.52) & 19.5 & 161 & (0.52) \\ \hline \hline
12 & 20.0 & 162 & (0.52) & 19.5 & 161 & (0.52) & 19.0 & 161 & (0.52) \\ \hline
}
\caption{Random distinguishers for Salsa20 ciphertext.}
\label{tab:estream-salsa}
\end{table}

\begin{table}[htb]
\centering
\resultsTable{
1--8 & 0.0${}^{*}$ & 0 & $n=104$ & 0.0${}^{*}$ & 0 & $n=101$ & 0.0${}^{*}$ & 0 & $n=104$ \\ \hline
9 & 1.0 & 1 & $n=234$ & 1.5 & 1 & $n=491$ & 2.0 & 1 & $n=121$ \\ \hline
10 & 2.0 & 13 & $n=188$ & 3.0 & 13 & $n=218$ & 3.0 & 12 & $n=158$ \\ \hline
11 & 10.0 & 157 & (0.52) & 11.5 & 157 & (0.52) & 14.0 & 159 & (0.52) \\ \hline
12 & 16.0 & 162 & (0.52) & 17.0 & 161 & (0.52) & 17.5 & 162 & (0.52) \\ \hline
13 & 20.0 & 162 & (0.52) & 20.0 & 162 & (0.52) & 19.0 & 162 & (0.52) \\ \hline \hline
32 & 20.0 & 161 & (0.52) & 20.0 & 162 & (0.52) & 20.0 & 161 & (0.52) \\ \hline
}
\caption{Random distinguishers for TSC-4 ciphertext.}
\label{tab:estream-tsc}
\end{table}

\chapter{Analysis of Salsa20 output stream}
\label{chap:analysis-salsa}
\begin{itemize}
\item learns current vectors quicker than other ciphers
\item the case of six
\end{itemize}

\chapter{Distinguishing hash outputs from random stream}
\label{chap:distinguish-hash}

\begin{itemize}
\item introduction, idea
\item hash function candidates from SHA-3
\end{itemize}

\section{Hash functions used}
\label{sec:hash-functions}

\begin{itemize}
\item except for 2 (?? source code size, compilation)
\item from last phase
\item those that could be limited in rounds are tested in weaker variant as well
\item differences from Ondrej Dubovec Bc thesis
\end{itemize}

\section{Generating binary stream from hash functions}
\label{sec:hash-settings}

\begin{itemize}
\item length set to 256b
\item hashing 4 byte counters starting from random value (in fact, cutting each hash in half)
\end{itemize}

\section{Determining optimal set change frequency}
\label{sec:hash-set-change-freqency}

\begin{itemize}
\item previously,we used change every 100 generations
\item 100 was taken from Matej Pristak's thesis
\item Ondrej proposes 10 as best, however, data is not provided
\item interpretation of results (\autoref{tab:hash-set-change-freqency}):
\begin{itemize}
\item ???
\end{itemize}
\end{itemize}

\begin{table}[htb]
\centering
\renewcommand{\arraystretch}{1.2}
\newcolumntype{C}{>{\centering\arraybackslash}X}
\begin{tabularx}{\textwidth}{|>{\raggedright\arraybackslash}p{2cm}||*{8}{C|}} \cline{2-9}
\multicolumn{1}{l||}{} & \multicolumn{8}{c|}{change frequency for test vector set} \\ \cline{2-9}
\multicolumn{1}{l||}{} & 5 & 10 & 20 & 50 & 100 & 200 & 500 & 1000 \\ \hline \hline
30\,000 g. & (0.614) & (0.614) & (0.607) & (0.602) & (0.599) & (0.598) & (0.591) & (0.582) \\ \hline
run-time & 70 m. & 52 m. & 42 m. & 37 m. & 32 m. & 28 m. & 23 m. & 20 m. \\ \hline \hline
300 sets & (0.567) & (0.583) & (0.585) & (0.589) & (0.599) & (0.608) & (0.617) & (0.618) \\ \hline
run-time & 4 m. & 6 m. & 9 m. & 19 m. & 32 m. & 57 m. & 115 m. 220 m. \\ \hline
\end{tabularx}
\renewcommand{\arraystretch}{1.0}
\caption{Determining optimal change frequency for test vector set.}
\label{tab:hash-set-change-freqency}
\end{table}

\section{Results interpretation}
\label{sec:hash-results}

\begin{itemize}
\item ???
\end{itemize}

\begin{table}[htb]
\centering
%\renewcommand{\arraystretch}{1.2}
\newcolumntype{C}{>{\centering\arraybackslash}X}
\begin{tabularx}{\textwidth}{|>{\raggedright\arraybackslash}p{2cm}||*{8}{C|}} \cline{2-9}
\multicolumn{1}{l||}{} & \multicolumn{8}{c|}{number of rounds} \\ \cline{2-9}
\multicolumn{1}{l||}{} & 0 & 1 & 2 & 3 & 4 & 5 & 6 & full \\ \hline \hline
ARIRANG & $n=694$ & $n=707$ & $n=467$ & $n=1071$ & (full) & -- & -- & (0.52) \\ \hline
Aurora & $n=5614$ & (0.75) & (0.78) !!! & (0.52) & -- & -- & -- & (0.52) \\ \hline
Blake & $n=474$ & (0.52) & -- & -- & -- & -- & -- & (0.52) \\ \hline
Blue Midnight Wish & (0.52) & -- & -- & -- & -- & -- & -- & (0.52) \\ \hline
Cheetah & $n=181$ & $n=574$ & $n=708$ & (0.90) !!! & & & & (0.52) \\ \hline
CHI & (0.52) & -- & -- & -- & -- & -- & -- & (0.52) \\ \hline
CRUNCH & $n=104$ & $n=534$ & $n=954$ & & 34-(0.52) & & & (0.52) \\ \hline
CubeHash & $n=104$ & (0.52) & -- & -- & -- & -- & -- & (0.52) \\ \hline
DCH & $n=104$ & (0.73) !!! & (0.52) & -- & -- & -- & -- & (0.52) \\ \hline
Dynamic SHA & $n=484$ & $n=2337$ & $n=1773$ !!! & (0.95) !!! & & & & (0.52) \\ \hline
Dynamic SHA2 & -- & (0.94) !!! & (0.74) & (0.75) & (0.57) & & & (0.52) \\ \hline
\end{tabularx}
%\renewcommand{\arraystretch}{1.0}
\caption{Random distinguishers for SHA-3 candidate functions.}
\label{tab:hash-distinguishers}
\end{table}

\chapter{Conclusions and future work}
\label{chap:conclusions}



\section{Conclusions based on experimental data}
\label{sec:outro-conclusions}

\begin{itemize}
\item summary of what we did
\item control distinguishers (random-random, hr-de, audio)
\item estream (round limited ciphers)
\item analysis of Salsa20
\item sha3 (round limited hash functions)
\end{itemize}

\begin{itemize}
\item different approach than statistical batteries -> possibly new things
\item dynamically adapting distinguisher - both advantage and disadvantage
\item comparable to statistical tests, however smaller inputs
\item speed: slow learning (more computational power needed), fast distinguishing
\item problem with interpreting results
\end{itemize}

\section{Proposed future work}
\label{sec:outro-future-work}

\begin{itemize}
\item deep analyses instead of wide
\item possibilities of longer input 
\begin{itemize}
\item READX
\item memory circuit
\end{itemize}
\item tools for interpreting results
\begin{itemize}
\item histogram of outputs in nodes
\end{itemize}
\item fixing functions in layers
\end{itemize}

\end{document}
