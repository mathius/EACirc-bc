\documentclass[12pt,oneside]{fithesis2}

% ===== LOADING PACKAGES =====
% language settings, main documnet language last
\usepackage[english]{babel}
% enabling new fonts support (nicer)
\usepackage{lmodern}
% setting input encoding
\usepackage[utf8]{inputenc}
% setting output encoding
\usepackage[T1]{fontenc}
% fithesis2 requires csquotes
\usepackage{csquotes}
% set page margins
\usepackage[top=3.5cm, bottom=3cm, left=2.4cm, right=2.4cm]{geometry}
% package to make bullet list nicer
\usepackage{enumitem}
% math symbols and environments
\usepackage{mathtools}
% packages for complex tables
\usepackage{tabularx}
\usepackage{multirow}
\usepackage{siunitx}
\usepackage{dcolumn}
\usepackage{array}
% generating hyperlinks in document
\usepackage{url}
\usepackage[plainpages=false,pdfpagelabels,colorlinks=true]{hyperref}

% ===== MAIN DOCUMENT SETTINGS =====
% adjusting hyphenation penalties
\tolerance=10000
\hyphenpenalty=500
% FI THESIS settings
\thesistitle{Usage of evolvable circuit \\for statistical testing \\of randomness}
\thesissubtitle{Bachelor thesis}
\thesisstudent{Martin Ukrop}
\thesiswoman{false}
\thesisfaculty{fi}
\thesisyear{spring 2013}
\thesisadvisor{RNDr.\ Petr Švenda,\ Ph.D.}
\thesislang{en}
% set correct spacing in itemize
\setitemize{noitemsep,topsep=0pt,parsep=0pt,partopsep=0pt}

% new commands for results table headers
\newcommand{\rotatedHeader}[2][l]{\rotatebox{90}{\begin{tabular}[#1]{@{}l}#2\end{tabular}}}
\newcommand{\resultsTable}[1]{%
\newcolumntype{C}{>{\centering\arraybackslash}X}%
\begin{tabularx}{\textwidth}{|r*{3}{||S[table-format=2.1]|S[table-format=3.0]|C}|} \hline
\multirow{1}*{\raisebox{-\height-0.3cm}{\rotatebox{90}{\# of rounds}}} & \multicolumn{9}{c|}{IV and key reinitialization} \\ \cline{2-10}
& \multicolumn{3}{c||}{once for run} & 
\multicolumn{3}{c||}{for each test set} & 
\multicolumn{3}{c|}{for each test vector} \\ \cline{2-10}
& \multicolumn{1}{c|}{\rotatedHeader{Dieharder\\(x/20)}} & \multicolumn{1}{c|}{\rotatedHeader{\textsc{Sts Nist}\\(x/162)}} &  \rotatedHeader{EACirc\\(\textsc{aam})}
& \multicolumn{1}{c|}{\rotatedHeader{Dieharder\\(x/20)}} & \multicolumn{1}{c|}{\rotatedHeader{\textsc{Sts Nist}\\(x/162)}} & \rotatedHeader{EACirc\\(\textsc{aam})}
& \multicolumn{1}{c|}{\rotatedHeader{Dieharder\\(x/20)}} & \multicolumn{1}{c|}{\rotatedHeader{\textsc{Sts Nist}\\(x/162)}} & \rotatedHeader{EACirc\\(\textsc{aam})} \\ \hline \hline
#1
\end{tabularx}}

% ===== BEGIN DOCUMENT =====
\begin{document}

\FrontMatter
\ThesisTitlePage

\begin{ThesisDeclaration}
\DeclarationText
\AdvisorName
\end{ThesisDeclaration}

\begin{ThesisThanks}
Thanks will be here.
\end{ThesisThanks}

\begin{ThesisAbstract}
Abstract will be here.
\end{ThesisAbstract}

\begin{ThesisKeyWords}
Keywords will be here.
\end{ThesisKeyWords}

\MainMatter
\tableofcontents
\chapter{Introduction}
\label{chap:intro}
Text ...

\chapter{Statistical randomness testing}
\label{chap:stat-rand-testing}

\begin{itemize}
\item idea: statistic (maths) -> test
\item fast
\item universal
\item usage: assess quality of (pseudo)random data, ???
\end{itemize}

\section{Statistical Test Suite by \textsc{Nist}}
\label{sec:sts-nist}

\begin{itemize}
\item nist standard
\item short description (?)
\end{itemize}

\section{Diehard battery of tests}
\label{sec:diehard}

\begin{itemize}
\item author
\item one of the first and most-well known in those years
\item old, but still considered "gold standard" along with sts-nist
\item short description of tests (?)
\end{itemize}

\section{Dieharder: A Random Number Test Suite}
\label{sec:dieharder}

\begin{itemize}
\item framework idea
\item progress: diehard -> sts-nist -> new
\end{itemize}

\section{Limits and disadvantages of statistical testing}
\label{sec:limits-stat-testing}

\begin{itemize}
\item idea -> test (idea is always the predecessor)
\item check only one specific property
\item can only rarely be adapted to specific situation
\item results interpretation (what is wrong?)
\end{itemize}

\chapter{Evolution based randomness testing}
\label{chap:evo-based-testing}
\begin{itemize}
\item general description of GA
\item idea behind EACirc
\item previous evolution of EACirc (SensorSim -> bc, mgr -> today)
\item capabilities of EACirc
\begin{itemize}
\item general object model (+picture)
\item separate modules for projects
\item separate modules for evaluators
\item guaranteed bit-reproducibility
\item computation recommencing (state, ...)
\item static checker for pregenerated test vectors
\end{itemize}
\item EACirc is wider project beyond the scope of this thesis, thus project evolution, some parts are being redesigned
\end{itemize}

\chapter{Experiment settings and results}
\label{chap:settings-results}

\begin{itemize}
\item general evolution settings
\item main goal: finding strong distinguisher (over 99\% for 50 consecutive generations)
\item displayed average stable generation across 30 independent runs \\
(stable = fitness over $99\%$ for at least next 50 test sets)
\item if none stable generation was found, average average maximum fitness after test vector change is displayed in parentheses.
\item statistical batteries STS-NIST and Dieharder for reference
\item 250 MB of data used, same files for Dieharder and STS-NIST
\item STS-NIST settings (lenghts, 2 test omitted)
\item each test run 100 times on 1\,000\,000 bits
\item some runs had problems with tests RandomExcursions and RandomExcursionsVariant, these tests are not included in the result
\item STS-NIST results interpretation (scores 0, 1)
\item Dieharder settings
\item test corresponding to original Diehard (except for Diehard sums test)
\item each test run once, length of the stream decided by test
\item Dieharder results interpretation (scores 0, 0.5, 1)
\item displayed number of tests passed out of total (pass=1, weak=0.5, fail=0)
\end{itemize}

\chapter{Control distinguishers}
\label{chap:distinguish-control}

\begin{itemize}
\item introduction (what are reference points? what result are "normal" and "random"?)
\item control distinguisher random-random
\item no stable generations found
\item average average maximum fitness after test vector change: 0.52
\item Dieharder: 20/20
\item STS NIST: 188/188
\item dependence on number of test set and test set change frequency
\item distinguishing Croatia from Germany
\begin{itemize}
\item 6 files of 5 MB from each source
\item read from beginning (initial offset = 0) => same data
\item looking for distinguisher for each pair
\end{itemize}
\end{itemize}

\begin{table}[htb]
\centering
\begin{tabular}{|c|r|*{6}{c|}} \cline{3-8}
\multicolumn{2}{c|}{} & \multicolumn{6}{c|}{test set size} \\ \cline{3-8}
\multicolumn{2}{c|}{} & 200 & 500 & 1000 & 2000 & 5000 & 10\,000 \\ \hline
\multirow{5}*{\rotatebox{90}{population }}
& 5 & - & - & (0.509) & - & - & - \\ \cline{2-8}
& 10 & - & - & (0.514) & - & - & - \\ \cline{2-8}
& 20 & (0.544) & (0.527) & (0.520) & (0.514) & (0.509) & (0.506) \\ \cline{2-8}
& 50 & - & - & (0.526) & - & - & - \\ \cline{2-8}
& 100 & - & - & (0.530) & - & - & - \\ \hline
\end{tabular}
\caption{Dependence of AAM on population size and test vector set size.}
\label{tab:random-set-size-change}
\end{table}

\chapter{Distinguishing cipher outputs from random stream}
\label{chap:distinguish-cipher}

\begin{itemize}
\item cipher modes (iv+key initialization frequency)
\item tables with results
\item case of LEX (not weakening the cipher, only making shorter output)
\item case of TSC (producing binary stream of 0 for 1-8 rounds) => problems in 3 Dieharder tests
\item conclusions
\begin{itemize}
\item more or less as statistical batteries
\item dieharder better in some case than STS-NIST (is newer and some tests are redesigned)
\item statistical tests has much more input data compared to EACirc
\item using evolved distinguisher is quick
\end{itemize}
\end{itemize}

\begin{table}[htb]
\centering
\resultsTable{
1 & 0.0 & 0 & $n=2681$ & 0.0 & 0 & (0.85) & 0.0 & 5 & $n=1431$ \\ \hline
2 & 0.5 & 0 & (0.54) & 1.0 & 0 & (0.54) & 15.5 & 146 & (0.52) \\ \hline
3 & 1.0 & 0 & (0.53) & 1.0 & 0 & (0.53) & 15.0 & 160 & (0.52) \\ \hline
4 & 3.5 & 79 & (0.52) & 3.0 & 78 & (0.52) & 20.0 & 160 & (0.52) \\ \hline
5 & 4.5 & 79 & (0.52) & 3.5 & 91 & (0.52) & 17.5 & 161 & (0.52) \\ \hline
6 & 19.0 & 158 & (0.52) & 19.0 & 159 & (0.52) & 18.0 & 162 & (0.52) \\ \hline
7 & 18.5 & 162 & (0.52) & 19.0 & 161 & (0.52) & 20.0 & 161 & (0.52) \\ \hline \hline
8 & 20.0 & 162 & (0.52) & 20.0 & 159 & (0.52) & 19.0 & 161 & (0.52) \\ \hline
}
\caption{Random distinguishers for Decim ciphertext.}
\label{tab:estream-decim}
\end{table}

\begin{table}[htb]
\centering
\resultsTable{
1 & 20.0 & 162 & (0.52) & 20.0 & 161 & (0.52) & 18.0 & 162 & (0.52) \\ \hline \hline
4 & 20.0 & 162 & (0.52) & 20.0 & 162 & (0.52) & 20.0 & 162 & (0.52) \\ \hline
}
\caption{Random distinguishers for FUBUKI ciphertext.}
\label{tab:estream-fubuki}
\end{table}

\begin{table}[htb]
\centering
\resultsTable{
1 & 0.0 & 0 & $n=221$ & 0.0 & 0 & (0.67) & 18.5 & 162 & (0.52) \\ \hline
2 & 0.0 & 0 & $n=471$ & 0.5 & 0 & (0.66) & 20.0 & 162 & (0.52) \\ \hline
3 & 19.5 & 160 & (0.52) & 20.0 & 162 & (0.52) & 20.0 & 162 & (0.52) \\ \hline \hline
13 & 20.0 & 162 & (0.52) & 20.0 & 161 & (0.52) & 19.5 & 162 & (0.52) \\ \hline
}
\caption{Random distinguishers for Grain ciphertext.}
\label{tab:estream-grain}
\end{table}

\begin{table}[htb]
\centering
\resultsTable{
1 & 20.0 & 162 & (0.52) & 20.0 & 162 & (0.52) & 20.0 & 162 & (0.52) \\ \hline \hline
10 & 20.0 & 160 & (0.52) & 20.0 & 162 & (0.52) & 20.0 & 162 & (0.52) \\ \hline
}
\caption{Random distinguishers for Hermes ciphertext.}
\label{tab:estream-hermes}
\end{table}

\begin{table}[htb]
\centering
\resultsTable{
1 & 0.0 & 0 & $n=148$ & 0.0 & 0 & $n=7274$ & 3.0 & 1 & $n=154$ \\ \hline
2 & 4.0 & 1 & $n=221$ & 4.0 & 1 & $n=304$ & 3.5 & 1 & $n=254$ \\ \hline
3 & 0.5 & 1 & $n=378$ & 3.5 & 1 & $n=491$ & 4.0 & 1 & $n=361$ \\ \hline
4 & 20.0 & 162 & (0.52) & 19.5 & 162 & (0.52) & 20.0 & 161 & (0.52) \\ \hline \hline
10 & 19.5 & 162 & (0.52) & 19.5 & 160 & (0.52) & 20.0 & 160 & (0.52) \\ \hline
}
\caption{Random distinguishers for LEX ciphertext.}
\label{tab:estream-lex}
\end{table}

\begin{table}[htb]
\centering
\resultsTable{
1 & 5.5 & 1 & (0.87) & 8.5 & 1 & (0.67) & 17.5 & 161 & (0.52) \\ \hline
2 & 5.5 & 1 & (0.87) & 7.0 & 1 & (0.67) & 19.5 & 162 & (0.52) \\ \hline
3 & 20.0 & 162 & (0.52) & 20.0 & 162 & (0.52) & 19.5 & 161 & (0.52) \\ \hline \hline
12 & 20.0 & 162 & (0.52) & 19.5 & 161 & (0.52) & 19.0 & 161 & (0.52) \\ \hline
}
\caption{Random distinguishers for Salsa20 ciphertext.}
\label{tab:estream-salsa}
\end{table}

\begin{table}[htb]
\centering
\resultsTable{
1 & 0.0${}^{*}$ & 0 & & 0.0${}^{*}$ & 0 & & 0.0${}^{*}$ & 0 & \\ \hline
2 & 0.0${}^{*}$ & 0 & & 0.0${}^{*}$ & 0 & & 0.0${}^{*}$ & 0 & \\ \hline
3 & 0.0${}^{*}$ & 0 & & 0.0${}^{*}$ & 0 & & 0.0${}^{*}$ & 0 & \\ \hline
4 & 0.0${}^{*}$ & 0 & & 0.0${}^{*}$ & 0 & & 0.0${}^{*}$ & 0 & \\ \hline
5 & 0.0${}^{*}$ & 0 & & 0.0${}^{*}$ & 0 & & 0.0${}^{*}$ & 0 & \\ \hline
6 & 0.0${}^{*}$ & 0 & & 0.0${}^{*}$ & 0 & & 0.0${}^{*}$ & 0 & \\ \hline
7 & 0.0${}^{*}$ & 0 & & 0.0${}^{*}$ & 0 & & 0.0${}^{*}$ & 0 & \\ \hline
8 & 0.0${}^{*}$ & 0 & & 0.0${}^{*}$ & 0 & & 0.0${}^{*}$ & 0 & \\ \hline
9 & 1.0 & 1 & $n=234$ & 1.5 & 1 & $n=491$ & 2.0 & 1 & $n=121$ \\ \hline
10 & 2.0 & 13 & $n=188$ & 3.0 & 13 & $n=218$ & 3.0 & 12 & $n=158$ \\ \hline
11 & 10.0 & 157 & (0.52) & 11.5 & 157 & (0.52) & 14.0 & 159 & (0.52) \\ \hline
12 & 16.0 & 162 & (0.52) & 17.0 & 161 & (0.52) & 17.5 & 162 & (0.52) \\ \hline
13 & 20.0 & 162 & (0.52) & 20.0 & 162 & (0.52) & 19.0 & 162 & (0.52) \\ \hline \hline
32 & 20.0 & 161 & (0.52) & 20.0 & 162 & (0.52) & 20.0 & 161 & (0.52) \\ \hline
}
\caption{Random distinguishers for TSC-4 ciphertext.}
\label{tab:estream-tsc}
\end{table}

\chapter{Analysis of Salsa20 output stream}
\label{chap:analysis-salsa}
\begin{itemize}
\item learns current vectors quicker than other ciphers
\item the case of six
\end{itemize}

\chapter{Distinguishing hash outputs from random stream}
\label{chap:distinguish-hash}
\begin{itemize}
\item hash function settings (hash length)
\item test vector generation method (4 byte counters starting from random point)
\item looking for best test set change frequency
\item tables with results
\item conclusions (???)
\end{itemize}

\chapter{Conclusions and future work}
\label{chap:conclusions}
\begin{itemize}
\item conclusions
\begin{itemize}
\item different approach than statistical batteries -> possibly new things
\item dynamically adapting distinguisher - both advantage and disadvantage
\item comparable to statistical tests, however smaller inputs
\item speed: slow learning (more computational power needed), fast distinguishing
\item problem with interpreting results
\item 
\end{itemize}
\item future work
\begin{itemize}
\item deep analyses instead of wide
\item possibilities of longer input 
\begin{itemize}
\item READX
\item memory circuit
\end{itemize}
\item tools for interpreting results
\begin{itemize}
\item histogram of outputs in nodes
\end{itemize}
\item fixing functions in layers
\end{itemize}
\end{itemize}

\end{document}
