\documentclass[12pt,oneside]{fithesis2}

% ===== LOADING PACKAGES =====
% language settings, main documnet language last
\usepackage[english]{babel}
% enabling new fonts support (nicer)
\usepackage{lmodern}
% setting input encoding
\usepackage[utf8]{inputenc}
% setting output encoding
\usepackage[T1]{fontenc}
% fithesis2 requires csquotes
\usepackage{csquotes}
% set page margins
\usepackage[top=3.5cm, bottom=3cm, left=2.4cm, right=2.4cm]{geometry}
% package to make bullet list nicer
\usepackage{enumitem}
% math symbols and environments
\usepackage{mathtools}
% packages for complex tables
\usepackage{tabularx}
\usepackage{multirow}
\usepackage{siunitx}
\usepackage{dcolumn}
\usepackage{array}
% enable page rotation
\usepackage{pdflscape}
% bibliography management
\usepackage[backend=biber, 		% use biber as backend instead of BiBTeX
	bibstyle=ieee-alphabetic, 	% bibliography style: IEEE with alphabetic citations
	citestyle=alphabetic, 		% citation style
	url=true, 					% display urls in bibliography
	hyperref=auto,				% detect hyperref and create links
	block=ragged 				% format bibliography into blocks, ragged on right
]{biblatex}
\addbibresource{thesis.bib}
% generating hyperlinks in document
\usepackage{url}
\usepackage[plainpages=false,pdfpagelabels,colorlinks=true]{hyperref}

% ===== MAIN DOCUMENT SETTINGS =====
% adjusting hyphenation penalties
\tolerance=10000
\hyphenpenalty=500
% space between paragraphs
\setlength{\parskip}{0.6em plus0.2em minus0.2em}
% set correct spacing in itemize
\setitemize{noitemsep,topsep=0pt,parsep=0pt,partopsep=0pt}
% FI THESIS settings
\thesistitle{Usage of evolvable circuit \\for statistical testing \\of randomness}
\thesissubtitle{Bachelor thesis}
\thesisstudent{Martin Ukrop}
\thesiswoman{false}
\thesisfaculty{fi}
\thesisyear{spring 2013}
\thesisadvisor{RNDr.\ Petr Švenda,\ Ph.D.}
\thesislang{en}

% new commands for results table headers
\newcommand{\m}[1]{\multirow{2}*{#1}}
\newcommand{\rotatedHeader}[2][l]{\rotatebox{90}{\begin{tabular}[#1]{@{}l}#2\end{tabular}}}
\newcommand{\resultsTable}[1]{%
\newcolumntype{C}{>{\centering\arraybackslash}X}%
\begin{tabularx}{\textwidth}{|r*{3}{||S[table-format=2.1]|S[table-format=3.0]|C}|} \hline
\multirow{1}*{\raisebox{-\height-0.3cm}{\rotatebox{90}{\# of rounds}}} & \multicolumn{9}{c|}{IV and key reinitialization} \\ \cline{2-10}
& \multicolumn{3}{c||}{once for run} & 
\multicolumn{3}{c||}{for each test set} & 
\multicolumn{3}{c|}{for each test vector} \\ \cline{2-10}
& \multicolumn{1}{c|}{\rotatedHeader{Dieharder\\(x/20)}} & \multicolumn{1}{c|}{\rotatedHeader{\textsc{Sts Nist}\\(x/162)}} &  \rotatedHeader{EACirc\\(\textsc{aam})}
& \multicolumn{1}{c|}{\rotatedHeader{Dieharder\\(x/20)}} & \multicolumn{1}{c|}{\rotatedHeader{\textsc{Sts Nist}\\(x/162)}} & \rotatedHeader{EACirc\\(\textsc{aam})}
& \multicolumn{1}{c|}{\rotatedHeader{Dieharder\\(x/20)}} & \multicolumn{1}{c|}{\rotatedHeader{\textsc{Sts Nist}\\(x/162)}} & \rotatedHeader{EACirc\\(\textsc{aam})} \\ \hline \hline
#1
\end{tabularx}}

% ===== BEGIN DOCUMENT =====
\begin{document}

\FrontMatter
\ThesisTitlePage

\begin{ThesisDeclaration}
\DeclarationText
\AdvisorName
\end{ThesisDeclaration}

\begin{ThesisThanks}
Thanks will be here.
\end{ThesisThanks}

\begin{ThesisAbstract}
Abstract will be here.
\end{ThesisAbstract}

\begin{ThesisKeyWords}
Keywords will be here.
\end{ThesisKeyWords}

\MainMatter
\tableofcontents
\chapter{Introduction}
\label{chap:intro}

\begin{itemize}
\item problem description
\item motivation -> EACirc
\item summary of experiments to follow
\item we = team of EACirc
\item work and research done by me, if not stated otherwise (but consulted within the team!)
\item licence of EACirc, licence of thesis text
\item fithesis2 used
\end{itemize}

\chapter{Statistical randomness testing}
\label{chap:stat-rand-testing}

The goal of randomness testing is to determine, whether the data provided is \textit{random}. 
The problem comes with the definition of randomness, since in truly random data, 
each fixed subsequence (e.\,g.\ sequence of a hundred zeroes) has the same probability of appearing.
Thus, statistical metrics have been developed to asses the matter of randomness.

All the statistical randomness tests are based on mathematical properties that hold for
\textit{most} of the random sequences with a sufficient length.
A simple example of such a property states that in each binary sequence the number of ones and zeroes should be 
approximately the same. It is crucial to be aware, that this will not hold for \textit{all} sequences (see the example above),
but the probability of randomly generating such a sequence sharply decreases with the increasing length.

Randomness testing based on statistical properties of data has both drawbacks and benefits, main of which are discussed below.
\begin{itemize} \rightskip=2em
\item \textbf{Speed}\\
Once the tests are implemented, they do not require excessive amount of time to perform -- 
the data is usually processed just once in a linear fashion.
\item \textbf{Universality}\\
Statistical tests can by applied to any binary data regardless of its origin -- they perform equally well. 
This can be viewed both as an advantage and disadvantage, since tests cannot be effortlessly adapted to specific situations.
\item \textbf{One-way design}\\
The creation of new test must be preceded by the idea and analysis of some useful statistical property. This part may be 
very complicated and usually requires a team of skilled mathematicians.
\item \textbf{Results interpretation}\\
The ever-present ambiguity in statistical measurements sometimes makes the results interpretation a highly non-trivial task.
It is crucial to understand what do the results indicate and what they do not. The above-mentioned finite sequence of binary zeroes
fails most of the statistical randomness tests, but its generation is just as probable 
as any other fixed binary sequence of the same length.
Put in another words, even the true random generator must produce non-random looking sequences once in a while.
\end{itemize}

\noindent
In practise, statistical randomness testing is being widely used in fields where the quality of random data is crucial, 
such as cryptography. To ease the assessment process, several statistical randomness testing suites have been developed, 
some of which are discussed below.

\section{Statistical Test Suite by NIST}
\label{sec:sts-nist}

Perhaps the most widely used battery of statistical tests is the Statistical Testing Suite 
by National Institute of Standards and Technology (STS NIST).
The primary motivations for developing this test suite was the need of standardised tests for detecting non-randomness 
in binary (pseudo)random sequences utilized in cryptographic applications. As well as designing the tests,
NIST provides their reference implementation and guidance in their use and application. \parencite{sts-nist}

The battery consists of 15 different tests, some of which can be run with several parameters. 
For detailed description of the tests, see the original documentation \parencite{sts-nist-documentation}. 
The implementation provided by NIST supports variable input data length and arbitrary number of independent data streams. 
The testing results provide the combined p-value of all data streams and the number of passed runs for each test 
according to the set significance level. 
Detailed setting used for the purposes of this thesis can be found in \autoref{sec:settings-statistics}.

\section{Diehard battery of tests}
\label{sec:diehard}

The second (unofficial) standard of statistical randomness testing is the Diehard Battery of Tests of Randomness, 
developed by George Marsaglia over several years at Florida State University. \parencite{diehard} 
Although now becoming slightly outdated, they were one of the first and most-well known 
in the pioneering years of statistical testing of randomness. 
For long, the Diehard Battery of Tests was considered a golden standard along with STS NIST.

The battery consist of 12 different tests. The original implementation, documentation and test descriptions are still available,
but since the code has not been revised from its creation in 1995, we chose not to use Marsaglia's original implementation.

\section{Dieharder: A Random Number Test Suite}
\label{sec:dieharder}

Dieharder, as its predecessors, aims to ease the testing of (pseudo)random generators and data for a variety purposes in research 
and cryptography. Developed by Robert G. Brown at the Duke University, it is designed to be as extensible as possible, 
allowing easy implementation of new tests and generators for testing. Most of the tests used allow for 
modifying the default parameters, enabling advanced users to fine-tune the testing process.
According to its creators, it is intended to be the ``Swiss army knife of random number test suite'', 
or if you prefer, ``the last suite you'll ever ware'' for testing random numbers. \parencite{dieharder}

After designing the testing framework, the development team gradually reimplemented and improved the original tests from 
the Diehard Battery of Tests of Randomness (see \autoref{sec:diehard}), 
the tests from STS NIST (see \autoref{sec:sts-nist}) and began to prepare and implement their own new tests.
The suite now contains 31 different tests from various sources. Tests can be run selectively.
The testing results provide the combined p-value for each test and a verdict of \textsc{passed}, \textsc{weak} or \textsc{failed}
according to the set significance levels.
Detailed settings used for the purposes of this thesis can be found in \autoref{sec:settings-statistics}.

\section{Drawbacks of human-designed statistical tests}
\label{sec:limits-stat-testing}

Although convenient in some ways, statistical randomness testing based on human-designed tests has several important drawbacks.
As mentioned above, the test creation must be preceded by an idea of mathematical property and its thorough analysis, 
which can be extremely time- and people-consuming. Further on, the tests are limited to one particular property and
adapting them to specific situation requires beginning the process of test creation all over again.

Both of the above-mentioned problems would be resolved if tests of comparable quality could be generated automatically, without 
the help of human specialists. Such concept and its comparison with human-generated tests is presented in the following chapters.

\chapter{Evolution-based randomness testing}
\label{chap:evo-based-testing}

In this chapter we try to describe a method of automatically generating statistical randomness tests. Compared to the standard
(manual) way of their creation, our approach would have a couple of advantages: 
\begin{itemize}
\item no prior knowledge of statistical properties of random data is needed;
\item test creation does not require excessive human analytical labour;
\item tests adapted for specific situations can be easily developed;
\item atypical and/or yet unknown data properties may be used.
\end{itemize}

\noindent
The main idea is to use supervised learning techniques based on evolutionary algorithms to design and further 
optimize a successful \textit{distinguisher} -- the test determining whether its input comes from a truly random source or not. 
The distinguisher will be represented as a hardware-like circuit consisting of a number of interconnected simple functions.
The evolution will use the principles of genetic programming.

\section{Basic principles of genetic programming}
\label{sec:basic-ga}

Genetic programming is a biologically inspired supervised learning technique. It tries to converge to optimal 
solution by making subtle changes to previous partial solutions, assessing their impact and propagating the perspective changes
until reaching the desired success rate. The existence of partial problem solutions is therefore essential.
The main flow of evolution implemented by genetic programming is as follows:
\begin{enumerate} \rightskip=2em
\item Firstly, a random set of partial solutions is generated. The solutions may be highly unsuccessful,
but some will nonetheless be better than others. This set of solutions is called a \textit{population}.
\item Secondly, the success of all individual solutions from the population is evaluated. The assessment is done using
a so called \textit{fitness function}. The quality of this function is crucial to the whole algorithm, as it
distinguishes the better and more successful partial solutions from the worse ones.
\item A new population of solutions is created by making a \textit{sexual crossover} of the best solutions from the 
previous generation. Informally put, solutions are subject to the survival of the fittest.
\item A small random change may be applied to some individuals in the new population. This \textit{mutation} prevents
the population from getting stuck in the local optimum and increases the chances of reaching a global optimum.
\item Steps 2-4 are iterated over and over, until the desired success rate of the population is achieved or the
required number of generations have evolved.
\end{enumerate}

\noindent
The principles of evolutionary algorithms induce a couple of design limitations and disadvantages. 
The most important ones include:
\begin{itemize} \rightskip=2em
\item Only problems with a sufficient space of partial solutions are applicable, since the individuals must be assessed 
to determine the fittest.
\item A small change in the solution should induce only a small change in the individual's fitness. If the changes were
too rapid, the evolution wouldn't be able to stabilize on the better and more successful solutions.
\item The evolution phase can be computationally very expensive, since making only small changes to the individuals may require
higher number of generations evolved.
\item It may be quite difficult to fine-tune the parameters (such as population size, mutation and crossover probabilities)
to achieve the best results.
\end{itemize}

\noindent
To counterweight the drawbacks, it must be noted, that evolutionary algorithms allow us to create solution not just for particular
instance of the problem, but to the whole set of similar problems -- we may be trying to evolve a universal solver, 
rather than for the solution itself. 
This improves the computation complexity, because after an expensive learning phase, the evolved solver may be used
repeatedly on multiple instances of the problem. However, the evolution of the general solver can be trickier than it seems,
since over-learning (i.\,e. finding the solution just to the particular instance of the problem) has to be avoided.

\section{Using software-emulated circuits}
\label{sec:sw-circuits}

Our goal is to create a simple circuit performing the desired task -- distinguishing the random and non-random data streams.
Thus, let's consider solutions in the form of of a hardware-like circuits with gates (\textit{function nodes}) 
and a set of wires (\textit{node connectors}).
Each node is responsible for computation of a simple function on its inputs (e.\,g.\ binary \textsc{and} operation).
Circuit nodes are positioned into layers, where outputs from one layer are connected to inputs of the next. Input of the whole
circuit is used as an input for the first layer and output of the last layer is considered the output of the entire circuit.
Connectors may only link adjacent layers, but may cross each other (contrary to real single-layer hardware circuits).
An example of such hardware-like circuit can be seen in \autoref{fig:circuit}.

In the current solution design, we consider only simple nodes operating on bytes. The supported functions are:
\begin{itemize}
\item common bit-manipulating functions (\textsc{or, and, xor, nor, nand, rotl, rotr, bitselector}),
\item simple arithmetical functions (\textsc{sum, subs, add, mult, div}),
\item identity function (\textsc{nop}) and
\item function reading specific input byte (\textsc{readx}).
\end{itemize}

\noindent
Although it would be sufficient to restrict ourselves to a smaller set of functions (e.\,g.\ \textsc{nand} only),
we chose to support a wider variety of functions as an human understandability trade-off.
More complex and sophisticated functions enable us to limit the circuit to significantly smaller number of layers and nodes,
while retaining a comparable expressive power.

To some extent, the structure of a software circuit resembles artificial neural networks 
(deep belief belief neural networks in particular). Notable differences are in
the learning mechanism and circuit dimensions (neural networks usually use very small number of layers). 
The function of individual nodes is different as well, since all nodes node in artificial neural networks perform the same function.

\section{EACirc: framework for automatic problem solving}
\label{sec:eacirc-principles}

Combining the principles of genetic programming and software circuits, we developed EACirc, the framework for automatic
problem solving. The initial version of EACirc was created by Petr Švenda at 
the Laboratory of Security and Applied Cryptography, Masaryk University and was loosely based on SensorSim,
locally developed application for simulation of sensor network \parencite{sensor-sim}.
This initial version provided the main shared functionality: evolutionary capabilities, software circuit emulation
and basic fitness evaluation. Later on, the application was improved by Matej Prišťák and Ondrej Dubovec 
(as their master and bachelor theses, respectively \parencite{thesis-pristak, thesis-dubovec}).

Afterwards, the object model of the entire project was redesigned and a handful of new features was added by myself. 
Most of the code that was taken over was revised and refactored as necessary
to ease the understanding of its function and to standardise naming and programming principles used throughout the project. 
Currently, the framework consist of the following main parts:
\begin{itemize}
\item \textbf{Evolutionary core}\\
The core evolutionary features are provided by GAlib, a C++ Library of Genetic Algorithm Components developed at MIT \parencite{galib}.
The library, when parametrized by function callbacks (e.\,g.\ function for mutation, sexual crossover, fitness function, \dots),
handles the main evolutionary actions.
\item \textbf{Circuit emulator}\\
The emulator simulates the behaviour of the circuit loaded from numerical representation. It plays a crucial role in
fitness assessment of the population.
\item \textbf{Project modules}\\
These modules are responsible for generating the data used in circuit fitness assessment. Each module (\textit{project}) corresponds
to one experiment (e.\,g.\ eStream candidate ciphers testing, SHA-3 candidate functions testing, \dots). The module's main
responsibility is to prepare the required number of problem--solution pairs in the form of circuit input stream (problem)
and optimal circuit output (solution). These pair are called a \textit{set of test vectors}.
\item \textbf{Evaluator modules}\\
Evaluator is a function responsible for yielding a numerical value of fitness, when provided with the pairs of
actual and expected circuit outputs. There are multiple approaches to evaluators -- the equality of expected and
actual output can be based on Hamming weight, numerical value, \dots
\item \textbf{Random generators}\\
Since evolutionary algorithms are highly randomized, a source of randomness is needed. To ensure the
computation determinism (all experiments need to be exactly reproducible), a hierarchy of random generators was developed.
To satisfy the varying needs, several generator types are implemented: true quantum random generator (based on pre-generated data),
configurable biased generator and low-entropy MD5-based generator.
\item \textbf{Self-tests}\\
For the ease of development, EACirc provides a handful of self-tests. Running these tests ensures the consistency
of seeding and data manipulation. Tests are implemented using CATCH, a C++ Automated Test Cases in Headers \parencite{catch}.
\item \textbf{XML manipulating library}\\
Most of the files produced and processed by the framework are XML-structured files. All these files are handles via
TinyXML, a simple, small, minimal, C++ XML parser library \parencite{tinyxml}.
\item \textbf{Static checker}\\
Although the static checker shares some code with the main framework, it is built as an independent application.
It is designed to verify obtained results (evolved circuits) by circumventing both the genetic manipulations and circuit emulator.
\item \textbf{Miscellaneous utilities}\\
EACirc framework comes with an assortment of scripts, used mainly for downloading, checking and processing the results.
\end{itemize}

\section{Current capabilities of EACirc}
\label{sec:eacirc-capabilities}

EACirc has a variety of other functions improving the core features of evolutionary algorithms and software circuit emulation.
This section provides a short and by no means exhaustive list of them.

\begin{itemize}
\item \textbf{Bit-reproducibility}\\
Bit-reproducibility is essential for the most research projects, since it enables replication and verification of the results.
EACirc uses genetic programming, which is fundamentally a randomized algorithm. Therefore, a hierarchy of random
generators with strictly defined scope of usage and seeding process was developed. This allowed us to replicate an experiment
by just providing the same input files and a fixed central seed.
\item \textbf{Computation recommencing}\\
After reaching bit-to-bit determinism, we implemented the ability to recommence older computations. 
To allow for this, EACirc was made capable of saving and loading its entire internal state to a set of XML-structured files.
This feature is especially useful for computation-expensive experiments -- when the machine is rebooted, we can continue from last saved state instead of starting all over again.
\item \textbf{Multi-format output}\\
For easy reusing and analysis, the evolved circuits are output in 4 different formats:
\begin{itemize}
\item binary output (useful for reloading the circuits into EACirc),
\item graph \textsc{dot} output (serves as a visual aid to human analyst),
\item simple text output (application-independent export format) and
\item program output (in the form of a stand-alone C program used for static analysis).
\end{itemize}
\noindent
The \textsc{dot} graph format can be easily displayed using the Graphviz library \parencite{graphviz} and thus 
facilitates manual analysis done by humans after the computation.\\
This functionality was implemented as early as the first version of EACirc. 
\item \textbf{Static checker for circuits in C}\\
Static checker is used to verify the success of evolved circuits exported as C programs. 
The verification uses pre-generated test vectors
and circumvents most parts of the EACirc framework, mainly the evolution and software circuit emulation.
The independence of this process is of utmost importance, since it provides supporting evidence for the achieved results.
\item \textbf{Modular object model}\\
When redesigning the object model, the principle of modules was utilized, thus enabling integration of multiple projects 
and evaluators according to actual needs. This greatly improved framework's flexibility and extensibility.
Currently, the following three projects (experiments) are implemented:
\begin{itemize}
\item Project for distinguishing between the output of eStream candidate ciphers and random stream of data was taken from the work of 
Matej Prišťák.
It was slightly revised to operate within the new object model and allow more detailed configuration.
\item Project for distinguishing between the output of SHA-3 candidate functions and random stream of data was inspired by the work of 
Ondrej Dubovec.
Hash functions implementations were taken over, but the test vector generation process was reimplemented from scratch. 
\item A small project for distinguishing among external binary files.
\end{itemize}
\end{itemize}

\noindent
Note, that EACirc is a project beyond the scope of this thesis. Some parts were added and/or redesigned in the process, so
different experiments may have incompatible configuration files and may have produced incomparable results.
For further details, user and development documentation, see EACirc wiki at GitHub \parencite{eacirc-github}.	

\chapter{Experiment settings and output data}
\label{chap:settings}

This chapter summarizes the configuration of EACirc used in the experiments presented in later chapters.
The accounts of random data used are given and EACirc outputs are described. 
In most experiments, our performance is compared to traditional batteries of statistical tests (STS NIST, Dieharder)
therefore settings and output description of these batteries is provided as well.

\section{EACirc settings}
\label{sec:settings-eacirc}

Most of the general settings (evolution and circuit parameters) were taken from Matej Prišťák's thesis \parencite{thesis-pristak}. 
The experiments supporting these parameters values were not reproduced, except for a few -- for details,
see \autoref{chap:distinguish-control}.

The evolution works with a population of 20 individuals, with a sexual crossover probability of 20\,\% and a mutation probability
set to 5\,\%. In each case (if not stated otherwise), we evolve 30\,000 generations with the test vector set
(learning data) changing every 100\textsuperscript{th} generation. Thus, a total of 300 unique test vector sets is used in each run.

The circuit dimensions are limited to 5 layers with a maximum of 8 function nodes per layer. It processes up to 16 input bytes
and produces 2 output bytes. Because of bad experience in previous work, using the \textsc{readx} function is forbidden.
All other implemented functions are allowed.

Each testing set consists of 1\,000 independent vectors, exactly half of which is truly random. 
According to research done by Matej Prišťák, the imbalance in test vectors would make the circuit learn what type is more frequent
in the particular set instead of developing a deterministic distinguisher. 
The order of random and non-random vectors in the set is not fixed.
Hence (\autoref{eq:eacirc-data-use}), all the results output by EACirc are based on a sample of about 2.5\,MB of assessed data.
\begin{equation}
\label{eq:eacirc-data-use}
\Sigma = \frac{30000 \text{ generations}}{100 \; \frac{\text{generations}}{\text{test set}}}
         \cdot \frac{1}{2} \cdot 1000 \; \frac{\text{vector}}{\text{test set}}
         \cdot 16 \; \frac{\text{bytes}}{\text{vector}}
         \approx 2.29 \text{\,MB}
\end{equation}

The expected circuit output is always \texttt{0x00} (zero byte) for a non-random vector and \texttt{0xff} (full byte) 
for a random one.
The used evaluator considers each of the output bytes separately, taking bytes with numerical interpretation lower than
128 as indicating a non-random stream and bytes higher than 127 as indicating a random stream.
Hence, the decision is based only on the first bit of each output byte.
Using the output of the evaluator, the fitness of the circuit is quantified as a quotient of a number of 
correctly predicted vectors and a total number of vectors in a set.

Experiment-specific settings (e.\,g.\ ways of generating non-random stream) are described in the 
appropriate chapters along with the results and their interpretation.

\section{Random data sources}
\label{sec:settings-random}

Eacirc requires a good source of randomness, since the the distinguishing process is based on comparing the assessed data
with a stream of data we declare to be random. All the achieved results therefore rise and fall 
on the quality of this referential stream.

Fortunately, quantum physics provides randomness with inherent unpredictability based on measuring quantum effects of photons. 
We acquired several hundred megabytes of quantum random data from the following on-line services:
\begin{itemize}
\item \textit{Quantum Random Bit Generation Service}\\
provided by Ruđer Bošković Institute in Zagreb, Croatia \parencite{qrng-service-croatia} and
\item \textit{High Bit Rate Quantum Random Number Generator Service}\\
provided by Humboldt University of Berlin, Germany \parencite{qrng-service-germany}.
\end{itemize}
The data from both sources have been thoroughly tested and compared, for details and results 
see \autoref{sec:control-germany-croatia}.

\section{EACirc output data}
\label{sec:settings-eacirc-output}

The randomized nature of evolutionary algorithms calls for multiple executions of each experiment due to variation in results.
For the most of the following experiments, we performed 30 independent runs. The final result presented is the average
of these 30 executions.

In each run, the maximum population success rate in the generations just after the change of test vectors are examined.
In our setting, this concerns the 1\textsuperscript{st}, 101\textsuperscript{st}, 201\textsuperscript{st}, \dots{} 
and 29901\textsuperscript{st} generation.
The presented results are of 2 types, depending on how good the found distinguishers are.
\begin{itemize}
\item If \textit{strong distinguishers} were found, we show the average number of generations needed
to reach them. For our purposes, a population of strong distinguishers has a maximum success rate in generations just
after the change of test vectors over 99\,\% during at least 50 consecutive test vector sets (5\,000 generations).
We call these distinguishers strong, because of their anticipated high success rate on streams they have not been learning on so far.
\item If a population of strong distinguishers was not reached during the evolution, 
we present the average value of maximal success rates in generations just after the change of test vectors,
further averaged across all 30 runs. This average average maximum (\textsc{aam}) is presented in parentheses.
\end{itemize}

\section{Settings and output data for statistical test batteries}
\label{sec:settings-statistics}

To compare our results with existing statistical tests, all experiments were replicated using standard batteries of statistical
randomness tests (STS NIST and Dieharder). 
For each setting in EACirc, an external file with 250\,MB of the assessed stream was created.
The same stream was used for both STS NIST and Dieharder tests. For further information on STS NIST and Dieharder, 
see \autoref{chap:stat-rand-testing}.

STS NIST was run on 100 sub-streams, each consisting of 1\,000\,000 bits. This amounts to about 11.92\,MB of assessed data.
All 15 available test were run in all supported configurations. Some runs had problems with tests \textit{Random Excursions} 
and \textit{Random Excursions Variant} (they considered no or less than 100 sub-streams during these test), 
so to ensure statistical accuracy of results, these test are omitted from the results.
For each test, the following results are output:
\begin{itemize}
\item the number of passed runs (a run is declared failed, if its p-value lies out of the interval determined by the significance
level of $\alpha = 0.01$) and
\item the combined p-value of all 100 runs of the test.
\end{itemize}
The result of all tests with all supported variants (162 tests in total, 2 tests excluded as mentioned above) 
is summarized in a cumulative score. The score assigns 1 to a test with both number of passed tests and the
combined p-value within the significance interval and assigning 0 otherwise. 
In summary, a fraction of 162/162 denotes a random stream (all tests passed) while a value of 0/162 denotes a highly non-random
stream (no test passed).

From the Dieharder suite, only the test corresponding to the original Diehard collection were used.
The only exception is the \textit{Diehard Sums Test} which was omitted, since the Dieharder community claims it has a couple of
implementation bugs and thus should not be used at all. Each of the chosen tests was rune just once, but was let
to process as much data as it required. Running the whole set processed about 582\,MB altogether with the smallest test
consuming about 3\,MB and the largest one about 127\,MB. Each test was labelled as \textsc{passed}, \textsc{weak} or \textsc{failed}
according to the threshold interval it falls within. The value of $\tau_{weak} = 0.005$ and $\tau_{fail} = 0.000001$ were used.
The result of the whole suite (20 tests in total) is again summarized in a cumulative score assigning 1 to a \textsc{passed} test,
0.5 to a \textsc{weak} test and 0 to a \textsc{failed} test.

\chapter{Control distinguishers}
\label{chap:distinguish-control}

Before performing the experiments themselves, we need to acquire reference results -- what does it mean
streams are indistinguishable from random in out context? Is the referential random data indistinguishable from random?
What is the \textsc{aam} value for the distinguishers?

\section{Looking for non-randomness in quantum random data}
\label{sec:control-random-random}

The first control experiment tries to distinguish quantum random data from other quantum random data.
We use 193\,MB of data obtained from Quantum Random Bit Generator Service (for details, see \autoref{sec:settings-random}).
We presume to fail at this and thus establish the randomness of the assessed data stream.

Using the standard statistical batteries confirmed out expectations -- all 20 tests of Dieharder passed as well as
all 162 tests of STS NIST. Running EACirc yielded the \textsc{aam} value of 0.52 with runs differing in 3\textsuperscript{rd}
or 4\textsuperscript{th} decimal place.

We anticipated that the difference of obtained \textsc{aam} from the naïve value of 0.50 was influenced by population size
and the amount of test vector in a set. This reasoning was based on the following two facts:
\begin{itemize}
\item \textsc{aam} value is based on fitness of the currently best individual in population (it's a maximum fitness) and thus
larger populations might have a better chance of getting a score above 0.50.
\item With the increasing number of test vectors in a set, the probability of just guessing correctly decreases.
\end{itemize}

\noindent
The performed experiments (\autoref{tab:random-set-size-change}) confirmed our presumptions: the \textsc{aam} value decreases
with decreasing population size and increasing size of test vector set. We can thus conclude that in our settings the \textsc{aam}
value of 0.52 corresponds to indistinguishable streams.

\begin{table}[h]
\centering
\renewcommand{\arraystretch}{1.2}
\newcolumntype{C}{>{\centering\arraybackslash}X}
\begin{tabularx}{\textwidth}{|c|r||*{6}{C|}} \cline{3-8}
\multicolumn{2}{c||}{} & \multicolumn{6}{c|}{number of test vector in a set} \\ \cline{3-8}
\multicolumn{2}{c||}{} & 200 & 500 & 1000 & 2000 & 5000 & 10\,000 \\ \hline \hline
\multirow{5}*{\rotatedHeader{individuals \\ in population}}
& 5 & -- & -- & (0.509) & - & - & - \\ \cline{2-8}
& 10 & -- & -- & (0.514) & - & - & - \\ \cline{2-8}
& 20 & (0.544) & (0.527) & (0.520) & (0.514) & (0.509) & (0.506) \\ \cline{2-8}
& 50 & - & - & (0.526) & - & - & - \\ \cline{2-8}
& 100 & - & - & (0.530) & - & - & - \\ \hline
\end{tabularx}
\renewcommand{\arraystretch}{1.0}
\caption{Dependence of AAM on population size and test vector set size.}
\label{tab:random-set-size-change}
\end{table}

\section{Distinguishing quantum random data from different sources}
\label{sec:control-germany-croatia}

Secondly, we want to compare quantum random data streams obtained from two different sources (for details, 
see \autoref{sec:settings-random}). We prepared 6 independent files of 5\,MB from each source, and attempted to find
a distinguisher for each pair. The initial reading offset was set to 0 in each of the files so that 
each file produced the same stream every time it was used.

For each pair, the computation was run just once (instead of 30 times). The results are summarized in 
\autoref{tab:control-germany-croatia} -- for each pair the average of the maximum population fitness in the generations
just after the test vector change is displayed (this would correspond to the \textsc{aam} value, if 30 runs
were performed for each pair).

The results oscillate closely around 0.52 indicating indistinguishable streams (see \autoref{sec:control-random-random}).
We can thus conclude that, for our purposes, both sources are equally random and equally reliable. 
Since no of the tested files expressed any statistically significant deviation from the others, we can use these files
interchangeably.

\begin{table}[h]
\centering
\renewcommand{\arraystretch}{1.2}
\newcolumntype{C}{>{\centering\arraybackslash}X}
\begin{tabularx}{\textwidth}{|c|r||*{6}{C|}} \cline{3-8}
\multicolumn{2}{c||}{} & \multicolumn{6}{c|}{QRBG service (Ruđer Bošković Institute, Croatia)} \\ \cline{3-8}
\multicolumn{2}{c||}{} & stream 1 & stream 2 & stream 3 & stream 4 & stream 5 & stream 6 \\ \hline \hline
\multirow{6}*{\rotatedHeader{QRNG service \\(HU, Germany)}}
& stream 1 & (0.521) & (0.520) & (0.520) & (0.519) & (0.519) & (0.519) \\ \cline{2-8}
& stream 2 & (0.518) & (0.519) & (0.520) & (0.520) & (0.520) & (0.519) \\ \cline{2-8}
& stream 3 & (0.519) & (0.522) & (0.519) & (0.520) & (0.519) & (0.519) \\ \cline{2-8}
& stream 4 & (0.520) & (0.520) & (0.519) & (0.518) & (0.519) & (0.519) \\ \cline{2-8}
& stream 5 & (0.519) & (0.520) & (0.519) & (0.518) & (0.520) & (0.520) \\ \cline{2-8}
& stream 6 & (0.520) & (0.519) & (0.520) & (0.520) & (0.519) & (0.519) \\ \hline
\end{tabularx}
\renewcommand{\arraystretch}{1.0}
\caption{Distinguishing binary quantum random streams from independent sources.}
\label{tab:control-germany-croatia}
\end{table}

\section{Uncompressed audio streams}
\label{sec:distinguishing-audio}

The third and last of the control experiments compares the set of audio files. We considered a set of 12 files -- 3 quantum random
data files, 3 uncompressed audio files with white, pink and Brownian noise, the same noise files with intermediate mp3 compression
and 3 samples of uncompressed black-metal music.

The quantum random data files had about 5\,MB and were turned into a listenable file by adding a \textsc{wav} header 
instructing to interpret the data as 2-channel, 16\,bit/sample, 44.1\,kHz PCM-encoded audio.
The 30 seconds (about 5.3\,MB) samples of white, pink and Brownian noise in the same audio format were generated using SoX
\parencite{sox}. The third subset was created from the above-mentioned generated noises by mp3 compression (bitrate of 
128\,kbps) and decompression back to the PCM-encoded audio. Note, that after compression the files
took about 480\,kB each (compared to 5.3\,MB of the uncompressed version).
The last three were 30 seconds samples of transcendental khaoblack metal by Abbey ov Thelema \parencite{abbey-ov-thelema}
all taken from the band's promo called \textit{MMXII: Here \& Now - At the Threshold ov End Times}.

We again attempted to develop a distinguisher for each pair of these files, again setting initial reading offset to 0
and performing just one execution per file combination. Our hypotheses included the following:
\begin{itemize}
\item the quantum random files will be indistinguishable from each other (assumption based on results from 
\autoref{sec:control-germany-croatia}),
\item the noise will be very similar to quantum random data (especially the white noise), but it may be able to be told apart,
\item the different noise types will be similar to each other,
\item the compressed and decompressed noise will be easily distinguishable from both the uncompressed noise and quantum random data
files (even though they cannot be easily differentiated by human ear),
\item the provided samples of metal music will be equally easily told apart from the other files, both uncompressed and re-compressed.
\end{itemize}

\noindent
The results are presented in the usual way in \autoref{tab:control-audio} (the part below the diagonal is mirrored to
facilitate analysis). From the analysis the the values, we accept most of the hypotheses:
\begin{itemize}
\item quantum random stream are undistinguishable (average maximum success rate of 0.52, see \autoref{sec:control-random-random}
for details),
\item generated white noise is completely undistinguishable from random data files,
\item pink and Brownian noise are easily told apart from each other or the quantum random files (success rate generally over 80\,\%),
\item mp3 compression has small, but detectable effect on the sound (although nearly undetectable by unskilled human ear, 
it successfully shifts the distinguisher success rate to about 0.58 when comparing with an uncompressed noise of the same kind),
\item used metal samples can be reliably distinguished from white noise (general success over 80\,\%), 
less so from pink and Brownian noise (success rate only around 65\,\%),
\item used metal samples are nearly indistinguishable from each other on the binary level 
(although the differences are easily detectable by human ear).
\end{itemize}

\begin{landscape}
\begin{table}[p]
\centering
\newsavebox{\temp}
\newcolumntype{C}{>{\centering \begin{lrbox}{\temp} \arraybackslash}X<{\end{lrbox} \m{\unhbox\temp} \arraybackslash}}
\begin{tabularx}{22cm}{|c|>{\raggedright\arraybackslash}p{2.5cm}*{4}{||C|C|C}|} \cline{3-14}

\multicolumn{2}{l||}{} & \multicolumn{3}{c||}{random streams} & \multicolumn{3}{c||}{noise (true)} &
\multicolumn{3}{c||}{noise (via mp3)} & \multicolumn{3}{c|}{metal music} \\ \cline{3-14}

\multicolumn{2}{l||}{} &  
\multicolumn{1}{c|}{\rotatedHeader{random\\stream 1}} & 
\multicolumn{1}{c|}{\rotatedHeader{random\\stream 2}} & 
\multicolumn{1}{c||}{\rotatedHeader{random\\stream 3}} & 
\multicolumn{1}{c|}{\rotatedHeader{white noise}} & 
\multicolumn{1}{c|}{\rotatedHeader{pink noise}} & 
\multicolumn{1}{c||}{\rotatedHeader{Brown noise}} & 
\multicolumn{1}{c|}{\rotatedHeader{white noise\\(via mp3)}} & 
\multicolumn{1}{c|}{\rotatedHeader{pink noise\\(via mp3)}} & 
\multicolumn{1}{c||}{\rotatedHeader{brown noise\\(via mp3)}} & 
\multicolumn{1}{c|}{\rotatedHeader{metal music\\(sample 1)}} & 
\multicolumn{1}{c|}{\rotatedHeader{metal music\\(sample 2)}} & 
\multicolumn{1}{c|}{\rotatedHeader{metal music\\(sample 3)}} \\ \cline{3-14} \hline \hline

\multirow{3}{*}[-20pt]{\rotatedHeader{random}} &
random stream 1 & 
n/a & (0.52) & (0.52) & (0.52) & (0.80) & (0.84) & (0.59) & (0.93) & (0.89) & (0.84) & (0.87) & (0.83) \\ \cline{2-14}
& random stream 2 &
(0.52) & n/a & (0.52) & (0.52) & (0.83) & (0.83) & (0.57) & (0.82) & (0.84) & (0.90) & (0.85) & (0.82) \\ \cline{2-14}
& random stream 3 & 
(0.52) & (0.52) & n/a & (0.52) & (0.94) & (0.91) & (0.58) & (0.83) & (0.83) & (0.89) & (0.83) & (0.85) \\ \hline \hline
\multirow{3}{*}[-10pt]{\rotatedHeader{noise (true)}} & 
white noise (true) &
(0.52) & (0.52) & (0.52) & n/a & (0.83) & (0.81) & (0.59) & (0.87) & (0.89) & (0.86) & (0.93) & (0.81) \\ \cline{2-14}
& pink noise (true) &
(0.80) & (0.83) & (0.94) & (0.83) & n/a & (0.76) & (0.86) & (0.52) & (0.76) & (0.65) & (0.65) & (0.66) \\ \cline{2-14}
& Brown noise (true) &
(0.84) & (0.83) & (0.91) & (0.81) & (0.76) & n/a & (0.86) & (0.76) & (0.56) & (0.71) & (0.69) & (0.68) \\ \hline \hline
\multirow{3}{*}[-10pt]{\rotatedHeader{noise (mp3)}} & 
white noise (via mp3) &
(0.59) & (0.57) & (0.58) & (0.59) & (0.86) & (0.86) & n/a & (0.91) & (0.83) & (0.84) & (0.80) & (0.78) \\ \cline{2-14}
& pink noise (via mp3) &
(0.93) & (0.82) & (0.83) & (0.87) & (0.52) & (0.76) & (0.91) & n/a & (0.78) & (0.63) & (0.68) & (0.70) \\ \cline{2-14}
& Brown noise (via mp3) &
(0.89) & (0.84) & (0.83) & (0.89) & (0.76) & (0.56) & (0.83) & (0.78) & n/a & (0.71) & (0.69) & (0.67) \\ \hline \hline
\multirow{3}{*}[-5pt]{\rotatedHeader{metal music}} & 
metal music (sample 1) &
(0.84) & (0.90) & (0.89) & (0.86) & (0.65) & (0.71) & (0.84) & (0.63) & (0.71) & n/a & (0.54) & (0.56) \\ \cline{2-14}
& metal music (sample 2) &
(0.87) & (0.85) & (0.83) & (0.93) & (0.65) & (0.69) & (0.80) & (0.68) & (0.69) & (0.54) & n/a & (0.53) \\ \cline{2-14}
& metal music (sample 3) &
(0.83) & (0.82) & (0.85) & (0.81) & (0.66) & (0.68) & (0.78) & (0.70) & (0.67) & (0.56) & (0.53) & n/a \\ \cline{1-14}
\end{tabularx}
\caption{Distinguishing random streams and uncompressed audio (noise, compressed noise, metal music).}
\label{tab:control-audio}
\end{table}
\end{landscape}

\chapter{Distinguishing cipher outputs from random stream}
\label{chap:distinguish-cipher}

\begin{itemize}
\item introduction, idea, running EACirc along with statistical batteries
\item stream ciphers from eStream competition
\end{itemize}

\section{Stream ciphers used}
\label{sec:estream-ciphers}

\begin{itemize}
\item ciphers except for ?? (why??)
\item from last phase
\item those that could be limited in rounds are tested in weaker variant as well
\item differences from Metej Pristak thesis
\end{itemize}

\section{Generating binary stream from stream ciphers}
\label{sec:estream-settings}

\begin{itemize}
\item cipher modes (iv+key initialization frequency)
\item case of LEX (not weakening the cipher, only making shorter output)
\item case of TSC (producing binary stream of 0 for 1-8 rounds) => problems in 3 Dieharder tests
\end{itemize}

\section{Results interpretation}
\label{sec:estream-results}

\begin{itemize}
\item ???
\item more or less as statistical batteries
\item dieharder better in some case than STS-NIST (is newer and some tests are redesigned)
\item statistical tests has much more input data compared to EACirc
\item using evolved distinguisher is quick
\end{itemize}

\begin{table}[htb]
\centering
\resultsTable{
1 & 0.0 & 0 & $n=2681$ & 0.0 & 0 & (0.85) & 0.0 & 5 & $n=1431$ \\ \hline
2 & 0.5 & 0 & (0.54) & 1.0 & 0 & (0.54) & 15.5 & 146 & (0.52) \\ \hline
3 & 1.0 & 0 & (0.53) & 1.0 & 0 & (0.53) & 15.0 & 160 & (0.52) \\ \hline
4 & 3.5 & 79 & (0.52) & 3.0 & 78 & (0.52) & 20.0 & 160 & (0.52) \\ \hline
5 & 4.5 & 79 & (0.52) & 3.5 & 91 & (0.52) & 17.5 & 161 & (0.52) \\ \hline
6 & 19.0 & 158 & (0.52) & 19.0 & 159 & (0.52) & 18.0 & 162 & (0.52) \\ \hline
7 & 18.5 & 162 & (0.52) & 19.0 & 161 & (0.52) & 20.0 & 161 & (0.52) \\ \hline \hline
8 & 20.0 & 162 & (0.52) & 20.0 & 159 & (0.52) & 19.0 & 161 & (0.52) \\ \hline
}
\caption{Random distinguishers for Decim ciphertext.}
\label{tab:estream-decim}
\end{table}

\begin{table}[htb]
\centering
\resultsTable{
1 & 20.0 & 162 & (0.52) & 20.0 & 161 & (0.52) & 18.0 & 162 & (0.52) \\ \hline \hline
4 & 20.0 & 162 & (0.52) & 20.0 & 162 & (0.52) & 20.0 & 162 & (0.52) \\ \hline
}
\caption{Random distinguishers for FUBUKI ciphertext.}
\label{tab:estream-fubuki}
\end{table}

\begin{table}[htb]
\centering
\resultsTable{
1 & 0.0 & 0 & $n=221$ & 0.0 & 0 & (0.67) & 18.5 & 162 & (0.52) \\ \hline
2 & 0.0 & 0 & $n=471$ & 0.5 & 0 & (0.66) & 20.0 & 162 & (0.52) \\ \hline
3 & 19.5 & 160 & (0.52) & 20.0 & 162 & (0.52) & 20.0 & 162 & (0.52) \\ \hline \hline
13 & 20.0 & 162 & (0.52) & 20.0 & 161 & (0.52) & 19.5 & 162 & (0.52) \\ \hline
}
\caption{Random distinguishers for Grain ciphertext.}
\label{tab:estream-grain}
\end{table}

\begin{table}[htb]
\centering
\resultsTable{
1 & 20.0 & 162 & (0.52) & 20.0 & 162 & (0.52) & 20.0 & 162 & (0.52) \\ \hline \hline
10 & 20.0 & 160 & (0.52) & 20.0 & 162 & (0.52) & 20.0 & 162 & (0.52) \\ \hline
}
\caption{Random distinguishers for Hermes ciphertext.}
\label{tab:estream-hermes}
\end{table}

\begin{table}[htb]
\centering
\resultsTable{
1 & 0.0 & 0 & $n=148$ & 0.0 & 0 & $n=7274$ & 3.0 & 1 & $n=154$ \\ \hline
2 & 4.0 & 1 & $n=221$ & 4.0 & 1 & $n=304$ & 3.5 & 1 & $n=254$ \\ \hline
3 & 0.5 & 1 & $n=378$ & 3.5 & 1 & $n=491$ & 4.0 & 1 & $n=361$ \\ \hline
4 & 20.0 & 162 & (0.52) & 19.5 & 162 & (0.52) & 20.0 & 161 & (0.52) \\ \hline \hline
10 & 19.5 & 162 & (0.52) & 19.5 & 160 & (0.52) & 20.0 & 160 & (0.52) \\ \hline
}
\caption{Random distinguishers for LEX ciphertext.}
\label{tab:estream-lex}
\end{table}

\begin{table}[htb]
\centering
\resultsTable{
1 & 5.5 & 1 & (0.87) & 8.5 & 1 & (0.67) & 17.5 & 161 & (0.52) \\ \hline
2 & 5.5 & 1 & (0.87) & 7.0 & 1 & (0.67) & 19.5 & 162 & (0.52) \\ \hline
3 & 20.0 & 162 & (0.52) & 20.0 & 162 & (0.52) & 19.5 & 161 & (0.52) \\ \hline \hline
12 & 20.0 & 162 & (0.52) & 19.5 & 161 & (0.52) & 19.0 & 161 & (0.52) \\ \hline
}
\caption{Random distinguishers for Salsa20 ciphertext.}
\label{tab:estream-salsa}
\end{table}

\begin{table}[htb]
\centering
\resultsTable{
1--8 & 0.0${}^{*}$ & 0 & $n=104$ & 0.0${}^{*}$ & 0 & $n=101$ & 0.0${}^{*}$ & 0 & $n=104$ \\ \hline
9 & 1.0 & 1 & $n=234$ & 1.5 & 1 & $n=491$ & 2.0 & 1 & $n=121$ \\ \hline
10 & 2.0 & 13 & $n=188$ & 3.0 & 13 & $n=218$ & 3.0 & 12 & $n=158$ \\ \hline
11 & 10.0 & 157 & (0.52) & 11.5 & 157 & (0.52) & 14.0 & 159 & (0.52) \\ \hline
12 & 16.0 & 162 & (0.52) & 17.0 & 161 & (0.52) & 17.5 & 162 & (0.52) \\ \hline
13 & 20.0 & 162 & (0.52) & 20.0 & 162 & (0.52) & 19.0 & 162 & (0.52) \\ \hline \hline
32 & 20.0 & 161 & (0.52) & 20.0 & 162 & (0.52) & 20.0 & 161 & (0.52) \\ \hline
}
\caption{Random distinguishers for TSC-4 ciphertext.}
\label{tab:estream-tsc}
\end{table}

\chapter{Analysis of Salsa20 output stream}
\label{chap:analysis-salsa}
\begin{itemize}
\item learns current vectors quicker than other ciphers
\item the case of six
\end{itemize}

\chapter{Distinguishing hash outputs from random stream}
\label{chap:distinguish-hash}

\begin{itemize}
\item introduction, idea
\item hash function candidates from SHA-3
\end{itemize}

\section{Hash functions used}
\label{sec:hash-functions}

\begin{itemize}
\item except for 2 (?? source code size, compilation)
\item from last phase
\item those that could be limited in rounds are tested in weaker variant as well
\item differences from Ondrej Dubovec Bc thesis
\end{itemize}

\section{Generating binary stream from hash functions}
\label{sec:hash-settings}

\begin{itemize}
\item length set to 256b
\item hashing 4 byte counters starting from random value (in fact, cutting each hash in half)
\end{itemize}

\section{Determining optimal set change frequency}
\label{sec:hash-set-change-freqency}

\begin{itemize}
\item previously,we used change every 100 generations
\item 100 was taken from Matej Pristak's thesis
\item Ondrej proposes 10 as best, however, data is not provided
\item interpretation of results (\autoref{tab:hash-set-change-freqency}):
\begin{itemize}
\item ???
\end{itemize}
\end{itemize}

\begin{table}[htb]
\centering
\renewcommand{\arraystretch}{1.2}
\newcolumntype{C}{>{\centering\arraybackslash}X}
\begin{tabularx}{\textwidth}{|>{\raggedright\arraybackslash}p{2cm}||*{8}{C|}} \cline{2-9}
\multicolumn{1}{l||}{} & \multicolumn{8}{c|}{change frequency for test vector set} \\ \cline{2-9}
\multicolumn{1}{l||}{} & 5 & 10 & 20 & 50 & 100 & 200 & 500 & 1000 \\ \hline \hline
30\,000 g. & (0.614) & (0.614) & (0.607) & (0.602) & (0.599) & (0.598) & (0.591) & (0.582) \\ \hline
run-time & 70 m. & 52 m. & 42 m. & 37 m. & 32 m. & 28 m. & 23 m. & 20 m. \\ \hline \hline
300 sets & (0.567) & (0.583) & (0.585) & (0.589) & (0.599) & (0.608) & (0.617) & (0.618) \\ \hline
run-time & 4 m. & 6 m. & 9 m. & 19 m. & 32 m. & 57 m. & 115 m. & 220 m. \\ \hline
\end{tabularx}
\renewcommand{\arraystretch}{1.0}
\caption{Determining optimal change frequency for test vector set.}
\label{tab:hash-set-change-freqency}
\end{table}

\section{Results interpretation}
\label{sec:hash-results}

\begin{itemize}
\item ???
\end{itemize}

\begin{table}[htb]
\centering
%\renewcommand{\arraystretch}{1.2}
\newcolumntype{C}{>{\centering\arraybackslash}X}
\begin{tabularx}{\textwidth}{|>{\raggedright\arraybackslash}p{2cm}||*{7}{C|}} \cline{2-8}
\multicolumn{1}{l||}{} & \multicolumn{7}{c|}{number of rounds} \\ \cline{2-8}
\multicolumn{1}{l||}{} & 0 & 1 & 2 & 3 & 4 & 5 & full \\ \hline \hline
ARIRANG & $n=694$ & $n=707$ & $n=467$ & $n=1071$ & (full) & -- & (0.52) \\ \hline
Aurora & $n=5614$ & (0.75) & (0.78) !!! & (0.52) & -- & -- & (0.52) \\ \hline
Blake & $n=474$ & (0.52) & -- & -- & -- & -- & (0.52) \\ \hline
Blue Midnight Wish & (0.52) & -- & -- & -- & -- & -- & (0.52) \\ \hline
Cheetah & $n=181$ & $n=574$ & $n=708$ & (0.90) !!! & (0.86)!!! & (0.52) & (0.52) \\ \hline
CHI & (0.52) & -- & -- & -- & -- & -- & (0.52) \\ \hline
CRUNCH & $n=104$ & $n=534$ & $n=954$ & 10-$n=1327$ & 17-$n=774$ & 34-(0.52) & (0.52) \\ \hline
CubeHash & $n=104$ & (0.52) & -- & -- & -- & -- & (0.52) \\ \hline
DCH & $n=104$ & (0.73) !!! & (0.52) & -- & -- & -- & (0.52) \\ \hline
Dynamic SHA & $n=484$ & $n=2337$ & $n=1773$ !!! & (0.95) !!! & (0.74) & (0.61) & (0.52) \\ \hline
Dynamic SHA & from 6 -> & (0.59) & (0.59) & & & & \\ \hline
Dynamic SHA2 & -- & (0.94) !!! & (0.74) & (0.75) & (0.57) & (0.60) & (0.52) \\ \hline
Dynamic SHA2 & from 6 -> & (0.60) & & & & & \\ \hline
\end{tabularx}
%\renewcommand{\arraystretch}{1.0}
\caption{Random distinguishers for SHA-3 candidate functions.}
\label{tab:hash-distinguishers}
\end{table}

\begin{table}[htb]
\centering
%\renewcommand{\arraystretch}{1.2}
\newcolumntype{C}{>{\centering\arraybackslash}X}
\begin{tabularx}{\textwidth}{|>{\raggedright\arraybackslash}p{2cm}||*{7}{C|}} \cline{2-8}
\multicolumn{1}{l||}{} & \multicolumn{7}{c|}{number of rounds} \\ \cline{2-8}
\multicolumn{1}{l||}{} & 0 & 1 & 2 & 3 & 4 & 5 & full \\ \hline \hline
ECHO & -- & (0.73) !!! & (0.52) & -- & -- & -- & (0.52) \\ \hline
ESSENCE & 8-(0.52) & -- & -- & -- & -- & -- & (0.52) \\ \hline
Fugue & (0.52) & -- & -- & -- & -- & -- & (0.52) \\ \hline
Grøstl & $n=8651$ !!! & (0.58) & & & & & (0.52) \\ \hline
Hamsi & $n=12408$ !!! & (0.52) & -- & -- & -- & -- & (0.52) \\ \hline
JH & $n=581$ & $n=4398$ & & & & & (0.52) \\ \hline
Lesamnta & $n=791$ & $n=568$ & & & & & (0.52) \\ \hline
Luffa & $n=604$ & $n=2011$ & & & & & (0.52) \\ \hline
MD6 & $n=101$ & $n=1281$ & & & & & (0.52) \\ \hline
Sarmal & (0.52) & -- & -- & -- & -- & -- & (0.52) \\ \hline
SHAvite-3 & (0.52) & -- & -- & -- & -- & -- & (0.52) \\ \hline
SIMD & $n=5428$ & (0.52) & -- & -- & -- & -- & (0.52) \\ \hline
Tangle & $n=714$ & $n=6868$ & & & & & (0.52) \\ \hline
Twister & $n=474$ & $n=717$ & & & & & (0.52) \\ \hline
Vortex & $n=104$ & -- & -- & -- & -- & -- & $n=1257$ \\ \hline
WaMM & $n=1171$ & $n=5631$ & & & & & (0.52) \\ \hline
Waterfall & (0.52) & -- & -- & -- & -- & -- & (0.52) \\ \hline
\end{tabularx}
%\renewcommand{\arraystretch}{1.0}
\caption{Random distinguishers for SHA-3 candidate functions.}
\label{tab:hash-distinguishers2}
\end{table}

\chapter{Conclusions and future work}
\label{chap:conclusions}



\section{Conclusions based on experimental data}
\label{sec:outro-conclusions}

\begin{itemize}
\item summary of what we did
\item control distinguishers (random-random, hr-de, audio)
\item estream (round limited ciphers)
\item analysis of Salsa20
\item sha3 (round limited hash functions)
\end{itemize}

\begin{itemize}
\item different approach than statistical batteries -> possibly new things
\item dynamically adapting distinguisher - both advantage and disadvantage
\item comparable to statistical tests, however smaller inputs
\item speed: slow learning (more computational power needed), fast distinguishing
\item problem with interpreting results
\end{itemize}

\section{Proposed future work}
\label{sec:outro-future-work}

\begin{itemize}
\item deep analyses instead of wide
\item possibilities of longer input 
\begin{itemize}
\item READX
\item memory circuit
\end{itemize}
\item tools for interpreting results
\begin{itemize}
\item histogram of outputs in nodes
\end{itemize}
\item fixing functions in layers
\end{itemize}

% include citations not cited specifically
\nocite{*}
% print complete bibliography
\printbibliography

\end{document}
